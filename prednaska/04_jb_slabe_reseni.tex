\section{Slabé řešení rovnice}
Uvažujme transportní rovnici na nekonečné oblasti $\Omega = \Real$ (pro $x$):
\begin{equation}
    \label{eq:point_transport_1d}
    \prtl_t u(t,x) + v\prtl_x u(t,x) =0 
\end{equation}
kde $v$ je konstantní rychlost. Řešením je posunutá počáteční podmínka $u_0(x)$:
\[
    u(t,x) = u_0(x-vt)
\]
pro libovolnou diferencovatelnou funkci $u_0$ máme:
\[
    \prtl_t u + v\prtl_x u=u_0'(x-vt)(-v) + vu_0'(x-vt)=0
\]

Zdá se logické, že by toto mělo platit pro libovolnou počáteční podmínku $u_0$, ale rovnice je formulována tak, že 
to platí jen pokud je $u_0$ diferencovatelná. Zkusme se vrátit k tomu jak jsme transportní rovnici odvodili. Pomocí Raynoldsovy věty jsme dostali:
\[
    \int_{\Omega_t} \prtl_t u + \div(v u) \d x= \int_{\Omega_t} \prtl_t u + v\prtl_x u \d x= 0
\]
pro libovolnou oblast $\Omega_t$. A bodovou rovnici \eqref{eq:point_transport_1d} jsme dostali za předpokladu, že vnitřek integrálu je spojitý, 
tedy $u$ je spojitě diferencovatelná v prostoru i čase. Tedy požadavek na diferencovatelnost je ve skutečnosti umělý a i bez něj platí:
\[
    \int_\Real \phi(\prtl_t u + \prtl_x(vu)) \d x = 0
\]
pro libovolnou hladkou funkci $\phi$ s kompaktním nosičem. Nosič funkce je množina, kde je funkce nenulová:
\[
    \supp \phi = \{ x,\ \phi(x) > 0\}
\]
a v našem případě jsou kompaktní všechny omezené a uzavřené intervaly. Jde tedy o to, že $\phi$ musí být \emph{směrem k nekonečnu} nulová.

Nyní však nevíme co je $\prtl_t u$ pokud je $u_0$ nespojitá, např. $u_0(x)=\sgn(x)$. Abychom se tohoto problému zbavili použijeme Greenovu větu 
(zde vlastně jen integraci per partes):

\[
    \int_\Real \int_\Real \phi(t,x)\big(\prtl_t u + \prtl_x(vu)\big) \d x \d t= 
    \int_\Real \int_\Real -\prtl_t \phi u - \prtl_x \phi vu \d x \d t = 0
\]
pro libovolnou hladkou funkci $\phi(t,x)$ s kompaktním nosičem na $\Real \times \Real$.
Tato rovnice již skutečně platí pro libovolnou integrovatelnou počáteční podmínku $u_0$.


\subsection{Slabé řešení pro eliptické rovnice}
\label{sec:weak_sol_elliptic_eq}

Budeme řešit rovnici
\begin{equation}
    \label{ModelEq}
   -\div(\tn K\grad u) + \div(\vc v u) =f + \sigma_f(u_f - u) \quad\text{na }\Omega.
\end{equation}
Rovnice popisuje ustálené rozložení teploty v médiu, které se pohybuje rychlostním polem $\vc v$ a má obecně anisotropní 
tensor teplené vodivosti $\tn K$. Tento tenzor zahrnuje jak difúzi tepla, tak např. disperzi, t.j. zvýšenou vodivost ve směru proudění.
Na levé straně máme postupně difúzní člen a konvektivní člen. Na pravé straně je hustota objemových zdrojů tepla $f$ a kontaktní zdroj tepla.
Kontaktní sdroj modeluje například přenos tepla z tělesa, které se dotýká kovového plátu. Zde je $u_f$ teplota tělesa a $\sigma_f\ge 0$
koeficient přestupu tepla z tělesa na plát. Tento člen však může modelovat také přenost tepla z horniny do podzemní vody.

Eliptickou rovnici na omezené oblasti $\Omega$ je třeba doplnit okrajovými podmínkami na celé hranici $\prtl \Omega$. Základní tři typy podmínek jsou:

{\bf Dirichletova okrajová podmínka.} 
\[
    u(\vc x) = u_d(\vc x)\text{ na }\Gamma_d
\]
předepisuje teplotu $u_d$ na části hranice $\Gamma_d$. Pevná teplota na hranici modeluje situaci, kdy se těleso $\Omega$ dokonale vodivě dotýká termostatu - 
tělesa z velkou teplenou kapacitou.

{\bf Neumannova okrajová podmínka.}
\[
    (-\tn K \grad u + \vc v u) \cdot \vc n = q_n\text{ na }\Gamma_n.
\]
Člen vlevo se nazývá normálový teplený tok, který je předepsán jako $q_n$. V teorii se obvykle uvažuje jako kladný tok ve směru ven z oblasti (vnější normála), 
nicmáně z důvodu konzistence s objemovými zdroji se v praxi používá raději opačná konvence. Fyzikálně relevantní je případ kdy je hranice pevná 
$\vc v = 0$ a teplený tok je dán např. výkonem topidla na hranici. 

{\bf Robinova (Newtonova) okrajová podmínka.}
\[
     (-\tn K \grad u + \vc v u) \cdot \vc n = \sigma_r(u - u_r)\text{ na }\Gamma_r. 
\]
Opět je releventní především případ $\vc v =0$, kdy podmínka modeluje realistický přenost tepla s koeficientem $\sigma_r\ge 0$ z tělesa o teplotě $u_r$.
Zde je opět $\vc n$ vnější normála, tedy vlevo je tok ven z oblasti, který je kladný pokud je $u > u_r$ což souhlasí se znaménkem na pravé straně.

Mimo tyto základní podmínky se v reálných úlohách objevují nejrůznější další podmínky. Například pokud na části hranice vtéká do oblasti 
voda o dané teplotě $U$, bude, půjde o Neumannovu podmínku s $\vc v\ne 0$ a 
$q_n=U\vc v \cdot \vc n$. Pro výtok z oblasti však potřebujeme podmínku $q_n=u\vc v \cdot \vc n$, což lze považovat za Robinovu podmínku s $u_r=0$ 
a $\sigma_r=\vc v \cdot\vc n$. Ovšem rychlost $\vc v$ nemusí být na hranici dopředu známa, může být výsledkem řešení nějaké rovnice proudění, proto 
je obvykle tuto podmínku chýpat jako zvláštní typ.
V praxi se také objevuje například kombinace Neumannovy a Robinovy okrajové podmínky:
\[
    (-\tn K \grad u + \vc v u)\cdot \vc n = \sigma_r(u - u_r) + q_n
\]
Z hlediska teorie lze obvykle tyto zvláštní případy popsat pomocí předchozích tří typů, ale jsou užitečné pro praxi.

Dále předpokládáme, že množiny $\Gamma_d$, $\Gamma_n$, a $\Gamma_r$ jsou navzájem diskjunktní (nemají průnik) a jejich sjednocení (respektive sjednocení jejich uzávěrů)
je hranice $\prtl \Omega$.


\subsection{Slabá formulace eliptické úlohy}
\label{sec:weak_form_elliptic_eq}

V této kapitole odvodíme slabou formulaci rovnice vedení tepla, spolu s aplikací klasických okrajových podmínek.

Prvně přenásobíme rovnici \eqref{ModelEq} libovolnou hladkou testovací funkcí $\phi(\vc x) \in C^\infty(\ol\Omega)$ (hladá až do hranice) a integrujeme přes $\Omega$:
\[
    \int_\Omega \phi\Big(-\div(\tn K\grad u-\vc v u\Big) \d \vc x = \int_\Omega \phi \Big(f + \sigma_f(u_f - u)\Big) \d \vc x.
\]
Dále v na levé straně použijeme Greenovu větu k přehození divergence na testovací funkci:
\begin{equation}
    \label{eq:after_green}
    \int_\Omega \grad \phi \cdot (\tn K\grad u-\vc v u) \d \vc x + \int_{\prtl \Omega} -\phi(\tn K\grad u-\vc v u)\cdot \vc n \d s 
    = \int_\Omega \phi \Big(f + \sigma_f(u_f - u)\Big) \d \vc x.
\end{equation}
Než přistoupíme k aplikaci okrajových podmínek, zamysleme se jaké vlastnosti musí mít funkce $u$, aby tato rovnice vůbec měla smysl. Pokud budeme předpokládat, že 
všechny ostatní parametry jsou hladké, musí mít funkce $u$ alespoň integrovatelné derivace ($\prtl_{x_i} u \in L_1(\Omega)$). Každopádně funkce $u$ nemůže být libovolná, 
ale patří do nějakého vektorového prostoru funkcí $H^1(\Omega)$, který si přesně zavedeme až později, ale již víme, že jeho funkce mají integrovatelnou derivaci.

Pro začátek předpokládejme pouze \emph{homogenní} Dirichletovu okrajovou podmínku $u_d =0$. Pak je řešení $u$ ve skutečnosti z podprostoru:
\[
    V_0 = \{ u\in H^1(\Omega), u(\vc x) = 0\text{ na } \Gamma_d \}
\]
Jelikož už známe hodnotu řešení na hranici $\Gamma_d$ nepotřebujeme \eqref{eq:after_green} na této části hranice a proto se můžeme omezit na testovací funkce, které 
jsou na $\Gamma_d$ nulové, t.j. 
\[
    \phi \in \mathcal D_0 = \{\phi \in C^\infty(\ol\Omega),\ \phi(\vc x) = 0\text{ na } \Gamma_d\}.
\]

Nyní rozdělíme hraniční integrál na integrály přes části hranice odpovídající jednotlivým typům okrajových podmínek:
Na $\Gamma_d$ je $\phi=0$ příslušná člen je tedy nulový:
\[
    \int_{\Gamma_d} -\phi(\tn K\grad u-\vc v u)\cdot \vc n \d s = 0
\]

Na $\Gamma_n$ je tok roven $q$, máme tedy:
\[
    \int_{\Gamma_n} -\phi(\tn K\grad u-\vc v u)\cdot \vc n \d s = \int_{\Gamma_n} \phi q \d s.
\]

Podobně známe tok na $\Gamma_r$:
\[
    \int_{\Gamma_r} -\phi(\tn K\grad u-\vc v u)\cdot \vc n \d s = \int_{\Gamma_n} \phi \sigma_r(u - u_r) \d s.
\]

Dostáváme tak slabou formulaci rovnice \eqref{ModelEq}. Slabým řešením bude každá funkce $u$ z prostoru $V_0$, který splňuje
\begin{align}
    \label{eq:weak_homo_dirich}
    A(u,\phi) &:= \int_\Omega (\tn K\grad u-\vc v u) \cdot \grad \phi   +   \sigma_f u \phi \d \vc x 
              + \int_{\Gamma_r} \sigma_r u \phi \d s \\
              &=  \int_\Omega \Big(f + \sigma_f u_f)\Big)\phi \d \vc x + \int_{\Gamma_r} \sigma_r u_r \phi \d s + \int_{\Gamma_n} -q \phi \d s =: \langle F, \phi \rangle
\end{align}
pro všechna $\phi \in \mathcal D$.

Nyní uvažujme případ s obecnou dirichletovou okrajovou podmínkou $u_d\ne 0$. Budeme předpokládat, že $u_d$ lze prodloužit dovnitř oblasti $\Omega$. Tedy, že existuje 
funkce $\tilde u_d$ z prostoru $H^1(\Omega)$ taková, že $\tilde u_d = u_d$ na $\Gamma_d$ a jinak je $\tilde u_d$ libovolná. Pak řešení s obecnou Dirichletovou podmínkou
lze napsat jako $u = \tilde u_d + u_0$, kde $u_0$ je na $\Gamma_d$ nulová funkce, t.j. $u_0$ je z prostoru $V_0$. Funkce $u_0$ představuje neznámou část řešení, 
kterou dostaneme řešením problému:
\[
    A(u_0, \phi) = <F, \phi> - A(\tilde u_d, \phi) = <\tilde F, \phi>
\]



\subsection{Odvození klasické formulace ze slabé}
Je třeba ověřit, že slabá formulace je ekvivalentní se silnou formulací v případě, že řešení je dostatečně hladké. Předpokládejme tedy, že $u\in C^2(\Omega)$.
Použijeme v \eqref{eq:weak_homo_dirich} testovací funkci s nosičem uvnitř $\Omega$, t.j. $\phi$ je nulová na hranici. Dále použijeme Greenovu větu, 
hraniční integrály budou nulové a dostaneme:
\begin{equation}
    \label{eq:green_interior}
    \int_\Omega -\div (\tn K\grad u-\vc v u) \phi \d \vc x = \int_\Omega \Big(f + \sigma_f (u_f - u)\Big)\phi \d \vc x.
\end{equation}
Odtud plyne splnění bodové rovnice \eqref{ModelEq} v každém bodě uvnitř $\Omega$.

Dále potřebujeme odvodit splnění okrajových podmínek. Dirichletova podmínka je splněna přímo, jelikož $u=\tilde u_d + u_0$ a $u_0$ je nulová na $\Gamma_d$.
Pro odvození Neumannovy a Robinovy okrajové podmínky uvažujeme libovolnou (hladkou) testovací funkci $\psi$ s nosičem uvnitř $\Gamma_{nr} =\Gamma_n \cup \Gamma_r$. 
Dále tuto funkci hladc prodloužíme dovnitř  $\Omega$ a dostaneme  (hladkou) testovací funkci $\phi_\epsilon \in \mathcal D_0$, která má na 
$\Gamma_{nr}$ hodnotu $\psi$ a je nenulová pouze na tenkém proužku do vzdálenosti $\epsilon$ od hranice.
Nyní v \eqref{eq:weak_homo_dirich} použijeme testovací funkci $\phi_\epsilon$ a použijeme Greenovu větu, dostaneme:
\[
    \int_\Omega X(u)\phi_\epsilon \d \vc x + \int_{\Gamma_{nr}} (\tn K\grad u-\vc v u)\cdot \vc n \phi_\epsilon \d s    
              =  \int_{\Gamma_r} \sigma_r (u_r -u) \phi_\epsilon \d s + \int_{\Gamma_n} -q \phi_\epsilon \d s 
\]
kde první člen obsahuje všechny členy z předchozí rovnice \eqref{eq:green_interior}. Nyní provedeme limitu $\epsilon \to 0$. V této limitě se první intergál 
bude blížit nule, jelikož $\phi_\epsilon$ je nenulová na množině velikosti $\epsilon \times \abs{\Gamma_nr}$, což konverguje k nule. Naproti tomu ve zbylých 
hraničních integrálech je $\phi_\epsilon = \psi$, což na $\epsilon$ nezávisí, tedy dostaneme:
\[
    \int_{\Gamma_{nr}} -(\tn K\grad u-\vc v u)\cdot \vc n \psi \d s  
              =  \int_{\Gamma_r} \sigma_r (u -u_r) \psi \d s + \int_{\Gamma_n} q \psi \d s 
\]
a odtud obě okrajové podmínky.


\subsection{Ekvivalentní minimalizační úloha}



% \section{Existence a jednoznačnost slabého řešení}
% Existence a jednoznačnost řešení je základní vlastnost dobrého matematického modelu reality. Přitom není možné argumentovat tím, že v realitě 
% se vždy systém nějak chová (existence) a nachází se vždy v jednom stavu (jednoznačnost). O matematickém modelu totiž nevíme zda realitu popisuje. 
% Otázka existence a jednoznačnosti je také podstatná pro numerické řešení rovnic. 
% 
% Pro velkou třídu lineárních PDR je otázka  existence a jednoznačnost řešení zodpovězena následující větou:
% \begin{theorem}
% Nechť $V$ je Hilbertův prostor.
% Nechť $A$ je lineární operátor z $V$ do duálu $V^\star$, který je 
% \begin{enumerate}
%  \item omezený, tj. 
%   \[
%      \langle A(v), \phi \rangle \le \norm{A}\norm{v}_V\norm{\phi}_V \quad \text{pro všechna }\phi\in V
%   \]
%   \item eliptický (positivně definitní), tj.
%   \[
%      \langle A(v), v \rangle \ge \alpha \norm{v}^2_V
%   \]
%   pro nějakou konstantu $\alpha > 0$.
% \end{enumerate}
% Pak pro každý omezený lineární funkcionál $F\in V^\star$, má rovnice 
% \[
%     A(v)=F
% \]
% právě jedno řešení $v \in V$ a platí
% \[
%    \norm{v}_V \le C \norm{F}_{V^\star}.
% \]
% \end{theorem}
% 
