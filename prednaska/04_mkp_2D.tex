%%%%%%%%%%%%%%%%%%%%%%%%%%%%

\section{MKP pro eliptické úlohy}
V předchozí kapitole jsme avedli koncept slabého řešení. Slabá formulace rovnic lépe odpovídá 
fyzikální podstatě zákanů zachování, ze kterých řada parciálních diferenciálních rovnic (PDR) vychází a
zejména umožňuje dokázat existenci slabého řešení pro širokou řadu lineárních rovnic, jak ukážeme později. 
Existence řešení však neznamená, že bychom toto řešení dokázali zapsat. Naopak pro většinu parciálních 
diferenciálních rovnic neumíme zapsat jejich řešení pomocí běžných funkcí (tzv. řešení v uzavřeném tvaru).
Proto se požívají různé metody přibližného řešení, kdy najdeme funkci, která rovnici nesplňuje přesně, 
ale zato ji umíme nalézt, vyhodnotit, zobrazit.

Pro jednoduchost budeme aproximativní řešení v této kapitole demonstrovat na jednodušší rovnici:

\begin{equation}
 -\div\big(\tn K \grad u\big) = f\quad \text{v }\Omega
\end{equation}
s Robinovou okrajovou podmínkou:
\[
   (-\tn K \grad u) \cdot \vc n = \sigma_R (u - u_R) \quad \text{na }\prtl\Omega.
\]

Slabá formulace hledá řešení v prostoru $V = H^1(\Omega)$ tedy jistý prostor funkcí se slabou 
první derivací. Řešení musí splňovat rovnici:
\begin{align}
    \label{eq:weak_poisson}
    a(u, v) &:= \int_\Omega (\tn K\grad u) \cdot \grad v \d x +\int_{\prtl\Omega} \sigma u v\d s\\
            & = \int_\Omega fv\d x + \int_{\prtl\Omega} \sigma u_R v \d s =: b(v)
\end{align}
pro všechny testovací funkce $ v\in V$.


\subsection{Galerkinova metoda}
(Viz. Johnson \cite{Johnson}, kapitola 2)

Metoda konečných prvků, kterou si podrobněji představíme v další kapitole, je speciálním případem
obecnější Galerkinovy approximační metody. Základní myšlenkou je ponechat ve slabé formulaci stejnou 
bilineární formu $a(u,v)$ a stejnou lineární formu $b(v)$, ale nahradit nekonečně dimenzionální prostor
řešení $V$  podprostorem $V_h \subset \subset V$ s konečnou dimenzí $\dim V_h = N$. Velmi podobně jako slabé 
řešení je definováno přibližné řešení $u_h \in V_h$, které splňuje rovnici
\begin{equation}
   \label{eq:weak_galerkin} 
   a(u_h, v_h) = b(v_h) \text{ pro všechny funkce }v_h\in V_h.
\end{equation}


Lineární prostor $V_h$ má konečnou bázi, tedy exisuje $N$ lineárně nezávislých funkcí 
$\phi_j(\vc x) \in V_h$, $j=1,\dots, N$,
a každý prvek tohoto prostoru 
pak lze napsat jako jejich lineární kombinaci:
\[
    w(\vc x) = \sum_{j=1}^N w_j \phi_j(\vc x).
\]
Koeficienty $w_i$ jednoznačně určují funkci $w$ a opačně pro každou funkci $w\in V_h$ existuje jednoznačný 
vektor koeficientů $(w_1, \dots, w_N)^\top$. Z toho plyne, že stejně můžeme zapsat i přibližné řešení:
\[ u_h(\vc x) = \sum_{j=1}^N u_j \phi_j(\vc x) \]
a pro jeho určení stačí definovat vektor $N$ koeficientů $\vc u=(u_1,...,u_N)^\top$. K tomu nám však stačí $N$ lineárně nezávislých 
testovacích funkcí daných bází. 
Řešíme tedy soustavu $N$ rovnic o $N$ neznámých:

\[
   a(\sum_{j=1}^N u_j \phi_j(\vc x), \phi_i(\vc x)) = b(\phi_i(\vc x)), \quad \text{pro }i=1, \dots, N,
\]
a protože je forma $a$ bilineární, můžeme součet a koeficienty přesunout ven, takže získáme:
\[
    \sum_{i=1}^N a(\phi_j, \phi_i) u_j = l(\phi_i), \quad \text{pro }i=1, \dots, N.
\]
Nalevo dostáváme prvky matice $\tn A$: $a_{ji} = a(\phi_j, \phi_i)$ a napravo prvky vektoru $\vc b$: $b_i = b(\phi_i)$.
Řešení Galerkinovy aproximační úlohy na konečně rozměrném podprostoru $V_h$, tedy rovnice \eqref{eq:weak_galerkin}, je proto
ekvivalentní s řešením soustavy lineárních rovnic:
\[
 \tn A \vc u = \vc b.
\]

V případě naší konkrétní rovnice \eqref{eq:weak_poisson} můžeme bilineární formu a lineární funkcionál explicitně vyjádřit. Bilineární forma $a(u,v)$ je dána jako:
\[
a(u,v) = \int_\Omega (\tn K \nabla u) \cdot \nabla v \, \mathrm{d}x + \int_{\partial \Omega} \sigma u v \, \mathrm{d}s.
\]
Lineární funkcionál $b(v)$ je pak:
\[
b(v) = \int_\Omega f v \, \mathrm{d}x + \int_{\partial \Omega} \sigma u_R v \, \mathrm{d}s.
\]
Galerkinova metoda spočívá v tom, že hledáme aproximaci $u_h \in V_h$, která splňuje:
\[
\int_\Omega (\tn K \nabla u_h) \cdot \nabla v_h \, \mathrm{d}x + \int_{\partial \Omega} \sigma u_h v_h \, \mathrm{d}s = \int_\Omega f v_h \, \mathrm{d}x + \int_{\partial \Omega} \sigma u_R v_h \, \mathrm{d}s \quad \text{pro všechna } v_h \in V_h.
\]
Tato soustava je pak základem pro sestavení konečněprvkové aproximace řešení naší úlohy.




\subsection{Výpočetní síť}
\def\elements{\mathcal T}
Oblast $\Omega \subset \Real^2$ předpokládáme polygonální 
(tedy její hranice je tvořena rovnými úseky). Tuto oblast rozdělíme na trojúhelníky (tzv. \emph{elementy})
$T \in \elements$ tak, aby dva trojúhelníky sousedily buďto celou společnou stranou nebo 
společným vrcholem. Množina $\elements$ obsahuje všechny elementy a nazývá se \emph{triangulace} oblasti $\Omega$, nebo také \emph{výpočetní síť}.
Vrcholy trojúhelníků nazýváme \emph{uzly} sítě a značíme je $\vc N_j$, $j=1, \dots, N$.

Ve 2D je možné pracovat také s rozdělením na čtyřúhelníky, nebo kombinaci trojúhelníků a čtyřúhelníků.
Ve 3D se používá dělení na tetrahedrohy (čtyřstěny), deformované krychle, ale též se používají prvky 
s různými kombinacemi trojúhelníkových a čtyřúhelníkových stěn, např. trojboké hranoly nebo pyramidy.

\subsection{Konečné prvky prvního řádu}
V metodě konečných prvků volíme specifický podprostor $V_h$ a specifické bázové funkce.
% 
%Chat todo: 
%- popsat prostor lineárních bázových funkcí: spojité, po částech  lineární P^1 na elementech
%- ortogonalita bázových funkcí vůči uzlům sítě
% 
% 
Jako aproximační prostor $V_h$ volíme prostor funkcí, které jsou spojité na celé oblasti $\Omega$
a na každém elementu $T$ jsou polynomy prvního stupně, tedy lineární funkce, viz. Obr. \ref{fig:base_2d_lin}. 
% Tento prostor se označuje jako prostor po částech lineárních funkcí $P^1$.
% 
Bázové funkce $\phi_j$ jsou definovány tak, že pro každý uzel $\vc N_i$ platí:
\[
\phi_j(\vc N_i) = \delta_{ij},
\]
kde $\delta_{ij}$ je Kroneckerovo delta. To znamená, že bázová funkce $\phi_j$ nabývá hodnoty 
1 v uzlu $\vc N_j$ a hodnoty 0 ve všech ostatních uzlech. Díky této vlastnosti jsou bázové funkce 
ortogonální vůči uzlům sítě, což umožňuje snadné sestavení a řešení soustavy rovnic.

Na každém trojúhelníkovém elementu jsou bázové funkce $\phi_j$ lineární a jsou jednoznačně
určeny výše uvedenou podmínkou.

\begin{figure}[h]
\centering
\includegraphics{base_2d_lin}
\caption{Příklad bázové funkce pro prostor po částech lineárních funkcí na 2D oblasti.}
\label{fig:base_2d_lin}
\end{figure}


\subsection{Asemblace lineárního problému}
Integrály vystupující ve vyhodnocení prvků matice  $\tn A$ a vektoru $\vc b$ v Galerkinově aproximaci
si rozdělíme na integrály přes jednotlivé elementy.

%Chat todo:
%- matematicky zapsat pro rovnici 'eq:weak_poisson'
%- vysvětlit koncept lokálních matic $\tn A_e$ a vektorů pravé strany $\vc b_e$
%- pro demonstraci zapsat část finální matice s barevně vyznačenými příspěvky od dvou sousedících lokálních matice

Pro rovnici \eqref{eq:weak_poisson} rozdělíme integrály přes celou oblast $\Omega$ a přes její hranici $\partial\Omega$ na součet 
integrálů přes jednotlivé elementy $T$ a jejich okraje na hranici oblasti, což je průnik hranice 
elementu s hranicí oblasti: $\prtl T \cap \prtl \Omega$.
\[
a(u, v) = \sum_{T\in \elements} \left( \int_{T} (\tn K \nabla u) \cdot \nabla v \, \mathrm{d}x  
+  \int_{\partial T \cap \partial \Omega} \sigma u v \, \mathrm{d}s \right).
\]
Pro vnitřní elementy je druhý integrál nulový.
Analogicky rozdělíme lineární funkcionál:
\[
b(v) = \sum_{T\in \elements} \left( \int_{T} f v \, \mathrm{d}x  + 
      \int_{\partial T \cap \partial \Omega} \sigma u_R v \, \mathrm{d}s \right).
\]

Koncept lokálních matic $\tn A_T$ a vektorů pravé strany $\vc b_T$ spočívá v tom, 
že pro každý element $T$ vypočítáme matici $\tn A_T$ s prvky:
% Indexy slozek davam dolu v souladu s obvyklou notaci.
\[
a_{T,ij} = \int_{T} (\tn K \nabla \phi_j) \cdot \nabla \phi_i \, \mathrm{d}x 
+ \int_{\partial T \cap \partial \Omega} \sigma \phi_j \phi_i \, \mathrm{d}s,
\]
a vektor $\vc b_T$ s prvky:
\[
b_{T,i} = \int_{T} f \phi_i \, \mathrm{d}x + 
\int_{\partial T \cap \partial \Omega} \sigma u_R \phi_i \, \mathrm{d}s.
\]
% Takto definovane matice ale nejsou lokalni ve smyslu velikosti - maji stejny rad jako globalni matice
Tyto lokální matice a vektory poté sestavíme do globální matice $\tn A$ a vektoru $\vc b$ 
podle toho, jak jsou bázové funkce $\phi_i$ a $\phi_j$ přiřazeny k uzlům. 

Pro demonstraci si představme dva sousedící elementy $K$ a $L$, které sdílejí společnou stranu s vrcholy 
$\vc N_i$, $\vc N_j$. Jejich lokální matice $\tn A_K$ a $\tn A_L$ přispívají k celkové globální matici 
$\tn A$ tak, že prvky odpovídající společným uzlům se sčítají. Například pro řádky a sloupce $i$ a $j$ 
sdílené oběma elementy přispívají ke globální matici oba elementy $K$ a $L$, ale případně i další elementy
s vrcholy $\vc N_i$ a $\vc N_j$:
\[
a_{ii} = a_{K,ii} + a_{L,ii} + \dots, \quad  a_{jj} = a_{K,jj} + a_{L,jj} + \dots, \quad
a_{ij} = a_{ji} = a_{K,ij} + a_{L,ij} + \dots.
\]
Tímto způsobem sestavíme globální matici $\tn A$ z lokálních matic
$A^K$ pro $K \in \elements$ je součtem všech lokálních příspěvků z jednotlivých elementů.



\subsection{Referenční element}
Pro vyhodnocení lokální matice a vektoru potřebujeme umět definované bázové funkce 
vyhodnotit. Vyhodnocovat integrály přes jednotlivé elementy přímo je však nepraktické, proto 
každý element $T$ zobrazíme pomocí \emph{afinní transformace} na \emph{referenční element} $\hat T$,
na kterém definujeme bázové funkce a jejich gradienty.

%Chat todo: 
%- vysvětlit affinní transformaci, značit referenční souřadnice a funkce  se stříškou, e.g. $\hat x$ 
%- definovat vrcholy referenčního elementu (trojůhelník)
%- definovat na něm bázové funkce konrétním vzorcem pro souřadnice $\hat x_1$, $\hat x_2$

Referenční trojúhelník $\hat{T}$ definujeme jako trojúhelník s vrcholy:
\eq{ \label{eq:hat_N}
\hat{\vc N}_1 = (0, 0), \quad \hat{\vc N}_2 = (1, 0), \quad \hat{\vc N}_3 = (0, 1).
}
Na referenčním elementu definujeme bázové funkce $\hat{\phi}_i(\hat{x})$ pro $i = 1, 2, 3$ následovně:
\begin{align*}
\hat{\phi}_1(\hat{\xx}) &= 1 - \hat{x}_1 - \hat{x}_2, \\
\hat{\phi}_2(\hat{\xx}) &= \hat{x}_1, \\
\hat{\phi}_3(\hat{\xx}) &= \hat{x}_2.
\end{align*}
Tyto funkce jsou lineární na referenčním trojúhelníku 
a splňují podmínku $\hat{\phi}_i(\hat{\vc N}_j) = \delta_{ij}$. 
Jelikož jsou bázové funkce lineární, jsou jejich gradienty konstantní:
\begin{align*}
\nabla_{\hat{x}} \hat{\phi}_1 &= \begin{pmatrix} -1 \\ -1 \end{pmatrix}, \\
\nabla_{\hat{x}} \hat{\phi}_2 &= \begin{pmatrix} 1 \\ 0 \end{pmatrix}, \\
\nabla_{\hat{x}} \hat{\phi}_3 &= \begin{pmatrix} 0 \\ 1 \end{pmatrix}.
\end{align*}

\subsection{Afinní transformace na referenční element}
Afinní transformace je lineární zobrazení, které mapuje referenční element $\hat{T}$ na 
libovolný skutečný element $T$ v síti. Afinní transformace musí být jednoznačně určena
vrcholy $T$, které si označíme $\vc N_i$, $\vc N_j$, $\vc N_k$. 

Označíme-li souřadnice v referenčním elementu symbolem 
$\hat{\xx} = (\hat{x}_1, \hat{x}_2)$ a souřadnice ve skutečném elementu $x = (x_1, x_2)$, 
pak afinní transformace má tvar:
\[
\xx = \vc F_T(\hat{\xx}) = \tn J_T \hat{\xx} + \vc j_T,
\]
kde $\tn J_T$ je {\it Jacobiho matice} transformace a $\vc j_T$ je vektor posunutí.
Jacobiho matice a posunutí nyní určíme z podmínek mapování uzlů:
% \[
%  F_T(\vc N_1) = \vc N_i, \quad F_T(\vc N_2) = \vc N_j, \quad F_T(\vc N_š) = \vc N_k.
% \]
\begin{align*}
\vc F_T(\hat{\vc N}_1) &= \vc N_1, \\
\vc F_T(\hat{\vc N}_2) &= \vc N_2, \\
\vc F_T(\hat{\vc N}_3) &= \vc N_3.
\end{align*}
Protože vrcholy referenčního trojúhelníka jsou definovány vztahem \eqref{eq:hat_N},
% \[
% \hat{N}_1 = (0, 0), \quad \hat{N}_2 = (1, 0), \quad \hat{N}_3 = (0, 1),
% \]
dostaneme:
\[
\begin{aligned}
\vc N_1 &= \tn J_T \begin{pmatrix} 0 \\ 0 \end{pmatrix} + \vc j_T = \vc j_T, \\
\vc N_2 &= \tn J_T \begin{pmatrix} 1 \\ 0 \end{pmatrix} + \vc j_T, \\
\vc N_3 &= \tn J_T \begin{pmatrix} 0 \\ 1 \end{pmatrix} + \vc j_T.
\end{aligned}
\]
Z první rovnice plyne, že vektor posunutí je $\vc j_T = \vc N_1$. Ze zbývajících rovnic pak vypočteme sloupce Jacobiho matice:
\[
\begin{aligned}
\tn J_T \begin{pmatrix} 1 \\ 0 \end{pmatrix} &= \vc N_2 - \vc N_1 = \vc e_1, \\
\tn J_T \begin{pmatrix} 0 \\ 1 \end{pmatrix} &= \vc N_3 - \vc N_1 = \vc e_2.
\end{aligned}
\]
Tedy Jacobiho matice je určena jako:
\[
\tn J_T = \begin{pmatrix} \vc e_1 & \vc e_2 \end{pmatrix},
\]
kde $\vc e_1$ a $\vc e_2$ jsou strany skutečného trojúhelníka $T$ vycházející z vrcholu $\vc N_1$.



\subsection{Věta o substituci ve vícerozměrném integrálu}

Pro transformaci integrálů z elementu $T$ na referenční element 
$\hat{T}$ využijeme obecnou větu o substituci. Pokud $\vc F$ je hladké 
zobrazení mapující body $\hat{\vc x} \in \hat X$ na body 
$\vc x\in X$, pak platí:
\[
\int_{X} f(\vc x) \d \vc x = \int_{\hat{X}} 
f(\vc F(\hat{\vc x})) 
\big|\frac{\d \vc x}{\d \hat{\vc x}}\big| \d\hat{\vc x}.
\]
Po substituci se nám v integrálu objevil navíc tzv. {\it Jakobián}, 
tj.  absolutní hodnota determinantu Jacobiho matice. 
Tato věta nám umožňuje převést integrály přes obecné elementy 
na integrály přes referenční element, 
což značně usnadňuje jejich vyhodnocení.

\subsection{Aplikace afinní transformace}
V případě afinního zobrazení $F_T$ je Jakobiho matice i příslušný Jakobián
konstantní na každém elementu. Pro integrál přes element $T$ pak 
substitucí dostaneme konkrétnější tvar:
\[
\int_{T} f(\vc x) \d \vc x = \int_{\hat{T}} 
f(\vc F_T(\hat{\vc x})) 
\big|\tn J_T\big| \d\hat{\vc x}. 
\]
Podobně vypadá substituce přes hranici, 
ale zde je třeba aplikovat Jacobiho matici na normálu.
\[
\int_{\prtl T} f(\vc x) \d \vc s = \int_{\prtl \hat{T}} 
f(\vc F_T(\hat{\vc x})) 
|\tn J_T \hat{\vc n}| \d\hat{\vc s}. 
\]

Afinní transformace $\vc F_T$ zobrazuje referenční element 
$\hat{T}$ na skutečný element $T$ a 
umožňuje nám vyjádřit bázové funkce a jejich gradienty 
ve skutečných souřadnicích:
\[
\phi_i(\vc x) = \hat{\phi}_i(\hat{\vc x}).
\]
Gradient bázových funkcí ve skutečných souřadnicích získáme 
pomocí transformace gradientů (viz kap. \ref{sec:trans_grad} níže):
\[
\nabla_x \phi_i(\vc x) = \tn J_T^{-\top} 
\nabla_{\hat{\vc x}} \hat{\phi}_i(\hat{\vc x}).
\]
Transformace gradientů je klíčová pro správné vyhodnocení 
členů obsahujících derivace v lokální matici. 
Gradient bázové funkce ve skutečných souřadnicích získáme:
\[
\nabla_{\vc x} \phi_i(\vc x) 
= \nabla_{\vc x} \hat\phi_i(\hat{\vc x}) 
=  \frac{\prtl \hat{\vc x}}{\prtl \vc x} \nabla_{\hat{\vc x}} \hat{\phi}_i(\hat{\vc x})
= \tn J^{-\top}_T \nabla_{\hat{\vc x}} \hat{\phi}_i(\hat{\vc x}).
\]
Tento vztah vyplývá z řetězového pravidla pro derivace 
a vlastností afinní transformace. 
Inverzní transponovaná Jacobiho matice $\tn J_T^{-\top}$ 
zajišťuje správné měřítko a orientaci gradientů při transformaci 
z referenčních na skutečné souřadnice.

\subsection{Transformace gradientů}\label{sec:trans_grad}


Transformace gradientů je klíčová pro správné vyhodnocení členů obsahujících derivace v lokální matici. Chceme vyjádřit gradient bázové funkce ve skutečných souřadnicích \(\nabla_{\vc x} \phi_i(\vc x)\) pomocí gradientu v referenčních souřadnicích \(\nabla_{\hat{\vc x}} \hat{\phi}_i(\hat{\vc x})\).

Víme, že \(\phi_i(\vc x) = \hat{\phi}_i(\hat{\vc x})\), 
kde \(\hat{\vc x}\) závisí na \(\vc x\) 
skrze inverzní afinní transformaci 
\(\hat{\vc x} = \tn J_T^{-1} (\vc x - \vc j_T)\). 
Použijeme řetězové pravidlo pro parciální derivace:
\begin{equation}
\label{eq:chain}
\frac{\partial \hat{\phi}_k(\hat{\vc x})}{\partial \hat{x}_j} 
=\frac{\partial \phi_k(x_1(\hat{\vc x}), x_2(\hat{\vc x}))}{\partial \hat{x}_j} 
= \frac{\partial \phi_k}{\prtl x_1}\frac{\prtl x_1}{\partial \hat{x}_j}
+ \frac{\partial \phi_k}{\prtl x_2}\frac{\prtl x_2}{\partial \hat{x}_j}.
\end{equation}
Protože afinní transformace je lineární, platí:
\[
\vc x = \tn J_T \hat{\vc x} + \vc j_T .
\]
Derivováním podle $\hat{\vc x}$ dostaneme:
\begin{equation}
    \label{eq:jac_mat}
    \frac{\prtl x_i}{\prtl \hat{x}_j} = J_{T, ij}.
\end{equation}
Matice $\tn J_T = \grad_{\hat{\vc x}} \vc x$
má tedy $x_i$ v řádcích a $\frac{\prtl}{\prtl \hat{x}_j}$ ve sloupcích.
Dosadíme do \eqref{eq:chain} a pozorujeme, že na pravé straně máme násobení 
matice a vektoru. Vektorový zápis \eqref{eq:chain} pak je:
\[
    (\grad_{\hat{\vc x}} \hat\phi_k)^\top =  (\grad_{\vc x} \phi_k)^\top
        \begin{pmatrix}
        \frac{\prtl x_1}{\prtl \hat x_1} & \frac{\prtl x_2}{\prtl \hat x_1} \\
        \frac{\prtl x_1}{\prtl \hat x_2} & \frac{\prtl x_2}{\prtl \hat x_2} \\
        \end{pmatrix}
     = (\grad_{\vc x} \phi_k)^\top \tn J_T,
\]
přičemž gradienty jsou zde řádkové vektory, a proto Jacobiho matici násobíme zleva.
Pro gradienty jako sloupcové vektory vzorec transponujeme a násobíme inverzní maticí, 
abychom dostali vzorec pro gradienty bázových funkcí vůči reálným souřadnicím $\vc x$:
\[
    \grad_{\vc x} \phi_k = \tn J_T^{-\top} \grad_{\hat{\vc x}} \hat\phi_k.
\]


\subsection{Vyhodnocení lokální matice a pravé strany}

Použijeme-li výše uvedené vztahy, můžeme vyjádřit prvky lokální matice $\tn A_T$ a vektoru $\vc b_T$ jako integrály přes referenční element:

\[
A_{T,ij} = \int_{\hat{T}} \left( \tn K \tn J_T^{-\top} \nabla_{\hat{\vc x}} \hat{\phi}_j(\hat{\vc x}) \right) 
         \cdot \left( \tn J_T^{-\top} \nabla_{\hat{\vc x}} \hat{\phi}_i(\hat{\vc x}) \right) |\det \tn J_T| 
         \d\hat{\vc x} 
+ \int_{\partial \hat{T} \cap \partial \hat{\Omega}} 
\sigma \hat{\phi}_j(\hat{\vc x}) \hat{\phi}_i(\hat{\vc x}) |\tn J_T \hat{\vc n}|
\, \d\hat{s},
\]

\[
b_{T,i} = \int_{\hat{T}} f(\tn F_T(\hat{\vc x})) \hat{\phi}_i(\hat{\vc x}) 
|\det \tn J_T| \d\hat{\vc x} + 
\int_{\partial \hat{T} \cap \partial \hat{\Omega}} 
\sigma u_R(\tn F_T(\hat{\vc x})) \hat{\phi}_i(\hat{\vc x}) |\tn J_T \hat{\vc n}| \d\hat{s}.
\]
Zde $|\tn J_T \hat{\vc n}|$ představuje měřítko délky při transformaci stran.



% \subsubsection{Praktické vyhodnocení integrálů}
% 
% Protože bázové funkce a jejich gradienty jsou na referenčním elementu známé a jednoduché, můžeme integrály vyhodnotit numericky, například pomocí kvadraturních metod. Pro lineární bázové funkce a konstantní koeficienty mohou být některé integrály vyhodnoceny i analyticky.
% 
% \subsubsection{Shrnutí}
% 
% Použitím afinní transformace a věty o substituci jsme převedli vyhodnocení lokálních matic a vektorů na jednotný referenční element. To značně zjednodušuje implementaci metody konečných prvků, neboť stačí jednou spočítat integrály na referenčním elementu a poté je pomocí transformace aplikovat na všechny elementy sítě.
% 
% 
% % \subsection{Vyhodnocení forem}
% % \begin{theorem}[o substituci)
% %     \int_\Omega f(x) \d \vc x = \int_{\hat \Omega} f(\vc F(\hat{\vc x})) \frac{\d \vc x}{\d \hat{\vc x}} \d \hat{\vc x}
% % \end{theorem}
% 
% 
% Demonstrujme vyhodnocení integrálů ve formách $a(\cdot, \cdot)$, $l(\cdot)$ na následujícím integrálu, 
% který vznikne slabou formulací Laplaceovy rovnice:
% \[
%  a(\phi_i, \phi_j) = \int_\Omega \grad \phi_i(\vc x) \cdot \grad \phi_j(\vc x) \d \vc x
% \]
% Po rozdělení integrace na elementy $K$ z triangulace $\mathcal T$:
% \[
%  a(\phi_i, \phi_j) =\sum_{K\in \mathcal T} \int_K  \grad \phi_i(\vc x) \cdot \grad \phi_j(\vc x) \d \vc x 
% \]
% Každý prvek sítě $K$ převedeme na referenční element $\hat K$ pomocí substituce 
% ${\vc x} = \tn P_K \hat{\vc x} + \vc q_K$, $\d \vc x = \abs{\tn P_K} \d \hat{ \vc x}$. 
% Při této transformaci se zachovávají hodnoty funkcí, t.j. 
% \[
%     \phi(\vc x) = \phi\big( \vc x(\hat{\vc x}) \big)=\hat \phi(\hat{\vc x}) 
% \]
% ale mění se hodnoty derivací:
% \[
%    \frac{\prtl \hat\phi(\hat{\vc x}) }{\prtl \hat{x}_i} = \frac{\prtl \phi(\vc x(\hat{\vc x})) }{\prtl \hat{x}_i} = \frac{\prtl \phi(\vc x)}{\prtl x_j}\frac{\prtl x_j}{\prtl \hat x_i}
% \]
% neboli pro gradient $\grad_{\hat{\vc x}} \hat \phi = \tn P_K \grad_{\vc x} \phi$. Po substituci dostaneme tedy integrál:
% 
% \[
%   a(\phi_i, \phi_j) = \sum_{K\in \mathcal T} \int_{\hat K} 
%         \big[ \tn P_K^{-1} \grad_{\hat{\vc x}} \hat\phi_i \big] \cdot \big[ \tn P_K^{-1} \grad_{\hat{ \vc x}} {\hat \phi}_j \big] \abs{\tn P_K} \d \hat{\vc x}.
% \]
% Pro transformaci složitějších než simplexových 
% elementů je transformace nelineární a tudíž matice $\tn P_K$ a vektor posunutí $\vc q_K$ jsou obecně závislé na $\vc x$ resp. $\hat{\vc x}$.
% Použitím numerické integrace (viz. dále) na náš integrál pak dostaneme:
% \[
%  a(\phi_i, \phi_j) = \sum_{K\in \mathcal T} \sum_{q\in Q} \vc D_i(\vc q) \cdot  \vc D_j(\vc q) J(\vc q)w_{\vc q},
% \]                   
% kde $\vc D_i(\vc q) = \tn P_K^{-1}(\vc q) \grad \hat \phi_i(\vc q)$.

\subsection{Numerická integrace}
Pro vyhodnocení integálů je výhodné použít numerickou integraci (kvadraturu). Pro polynomiální bázové funkce lze zvolit takovou kvadraturu, že jsou 
příslušné integrály spočteny přesně. Pokud jsou pod integrálem i jiné funkce, například vlivem koeficientů v rovnici nebo kvůli 
transformaci na referenční element, poskytuje numerická kvadratura obvykle dobrou aproximaci. Konkrétní kvadratura na referenčním prvku $\hat T$ je dána
\begin{enumerate}
 \item množinou kvadraturních bodů $Q=\{\vc q \in \hat T\}$,
 \item množinou jim příslušejících vah $w_{\vc q}$.
\end{enumerate}
Integrál z funkce $f(\hat{\vc x})$ je pak aproximován pomocí sumy:
\[
    \int_{\hat T} f(\hat{\vc x}) \d \hat{\vc x} \approx \sum_{\vc q\in Q} f(\hat{\vc q}) w_{\vc q}.
\]

Pro integraci na reálném intervalu $T\subset \Real$ lze použít {\it Gaussovu kvadraturu},  která má optimální řád chyby. Při použití $n$ kvadraturních bodů integruje
kvadratura přesně polynomy do řádu $2n-1$. Pro vyhodnocení bilineární formy pro Laplaceovu rovnici a při použití prvků s polynomy řádu $k$ mají derivace řád $k-1$ a 
jejich součin řád $2(k-1)$. Pro přesné vyhodnocení integrálů je tedy potřeba použít $k$-bodovou Gaussovu kvadraturu. U 2D a 3D prvků je situace komplikovanější,
nicméně vhodné kvadratury existují pro všechny základní typy elementů.









% \section{MKP pro eliptické úlohy}
% V předchozí kapitole jsme uvedli koncept slabého řešení. Slabá formulace rovnic lépe odpovídá 
% fyzikální podstatě zákonů zachování, ze kterých řada parciálních diferenciálních rovnic (PDR) vychází a
% zejména umožňuje dokázat existenci slabého řešení pro širokou řadu lineárních rovnic, jak ukážeme později. 
% Existence řešení však neznamená, že bychom toto řešení dokázali zapsat. Naopak pro většinu parciálních 
% diferenciálních rovnic neumíme zapsat jejich řešení pomocí běžných funkcí (tzv. řešení v uzavřeném tvaru).
% Proto se používají různé metody přibližného řešení, kdy najdeme funkci, která rovnici nesplňuje přesně, 
% ale zato ji umíme nalézt, vyhodnotit, zobrazit.
% 
% Pro jednoduchost budeme aproximativní řešení v této kapitole demonstrovat na jednodušší rovnici:
% 
% \begin{equation}
%  -\div\big(\tn K \grad u\big) = f\quad \text{v }\Omega
% \end{equation}
% s Robinovou okrajovou podmínkou:
% \[
%    (-\tn K \grad u) \cdot \vc n = \sigma_R (u - u_R) \quad \text{na }\prtl\Omega
% \]
% 
% Slabá formulace hledá řešení v prostoru $V = H^1(\Omega)$, tedy jistý prostor funkcí se slabou 
% první derivací. Řešení musí splňovat rovnici:
% \begin{align}
%     \label{eq:weak_poisson}
%     A(u, v) &:= \int_\Omega (\tn K\grad u) \cdot \grad v \, \mathrm{d}x +\int_{\prtl\Omega} \sigma u v\, \mathrm{d}s\\
%             & = \int_\Omega f v\, \mathrm{d}x + \int_{\prtl\Omega} \sigma u_R v\, \mathrm{d}s =: l(v)
% \end{align}
% pro všechny testovací funkce $ v\in V$.
% 
% V případě naší konkrétní rovnice \eqref{eq:weak_poisson} můžeme bilineární formu a lineární funkcionál explicitně vyjádřit. Bilineární forma $A(u,v)$ je dána jako:
% 
% \[
% A(u,v) = \int_\Omega (\tn K \nabla u) \cdot \nabla v \, \mathrm{d}x + \int_{\partial \Omega} \sigma u v \, \mathrm{d}s.
% \]
% 
% Lineární funkcionál $l(v)$ je pak:
% 
% \[
% l(v) = \int_\Omega f v \, \mathrm{d}x + \int_{\partial \Omega} \sigma u_R v \, \mathrm{d}s.
% \]
% 
% Galerkinova metoda spočívá v tom, že hledáme aproximaci $u_n \in V_n$, která splňuje:
% 
% \[
% \int_\Omega (\tn K \nabla u_n) \cdot \nabla v \, \mathrm{d}x + \int_{\partial \Omega} \sigma u_n v \, \mathrm{d}s = \int_\Omega f v \, \mathrm{d}x + \int_{\partial \Omega} \sigma u_R v \, \mathrm{d}s \quad \text{pro všechna } v \in V_n.
% \]
% 
% Tato soustava je pak základem pro sestavení konečněprvkové aproximace řešení naší úlohy.
% 
% 
% \section{Galerkinova metoda}
% (Viz. Johnson \cite{Johnson}, kapitola 2)
% 
% Metoda konečných prvků, kterou si podrobněji představíme v další kapitole, je speciálním případem
% obecnější Galerkinovy aproximační metody. Základní myšlenkou je ponechat ve slabé formulaci stejnou 
% bilineární formu $A(u,v)$ a stejnou lineární formu $l(v)$, ale nahradit nekonečně dimenzionální prostor
% řešení $V$ podprostorem $V_n \subset V$ s konečnou dimenzí $\dim V_n = n$. Velmi podobně jako slabé 
% řešení je definováno přibližné řešení $u_n \in V_n$, které splňuje rovnici
% \begin{equation}
%    \label{eq:weak_galerkin} 
%    A(u_n, v) = l(v) \quad \text{pro všechny funkce }v\in V_n.
% \end{equation}
% 
% 
% Lineární prostor $V_n$ má konečnou bázi, tedy existuje $n$ lineárně nezávislých funkcí 
% $\phi_j(\vc x) \in V_n$, $j=1,\dots, n$,
% a každý prvek prostoru 
% pak lze napsat jako jejich lineární kombinaci:
% \[
%     w(\vc x) = \sum_{j=1}^n w_j \phi_j(\vc x).
% \]
% Koeficienty $w_j$ jednoznačně určují funkci $w$ a opačně pro každou funkci $w\in V_n$ existuje jednoznačný 
% vektor koeficientů $(w_1, \dots, w_n)$. Z toho plyne, že stejně můžeme zapsat i přibližné řešení
% a pro jeho určení stačí určit vektor $n$ koeficientů. K tomu nám však stačí $n$ lineárně nezávislých 
% testovacích funkcí daných bází. 
% Řešíme tedy soustavu $n$ rovnic o $n$ neznámých:
% 
% \[
%    A\left(\sum_{j=1}^n u_j \phi_j(\vc x), \phi_i(\vc x)\right) = l(\phi_i), \quad \text{pro }i=1, \dots, n.
% \]
% A protože je forma $A$ bilineární, můžeme součet a koeficienty přesunout ven a získáme:
% \[
%     \sum_{j=1}^n u_j A(\phi_j, \phi_i) = l(\phi_i), \quad \text{pro }i=1, \dots, n.
% \]
% Nalevo dostáváme prvky matice $A_{i,j} = A(\phi_j, \phi_i)$ a napravo prvky vektoru $b_i = l(\phi_i)$.
% Řešení slabé formulace na konečně rozměrném podprostoru, tedy rovnice \eqref{eq:weak_galerkin}, je tedy 
% ekvivalentní řešení lineární soustavy rovnic:
% \[
%  \tn A \vc u = \vc b.
% \]
% 
% V případě naší konkrétní rovnice \eqref{eq:weak_poisson} můžeme bilineární formu a lineární funkcionál explicitně vyjádřit. Bilineární forma $A(u,v)$ je dána jako:
% 
% \[
% A(u,v) = \int_\Omega (\tn K \nabla u) \cdot \nabla v \, \mathrm{d}x + \int_{\partial \Omega} \sigma u v \, \mathrm{d}s.
% \]
% 
% Lineární funkcionál $l(v)$ je pak:
% 
% \[
% l(v) = \int_\Omega f v \, \mathrm{d}x + \int_{\partial \Omega} \sigma u_R v \, \mathrm{d}s.
% \]
% 
% Galerkinova metoda spočívá v tom, že hledáme aproximaci $u_n \in V_n$, která splňuje:
% 
% \[
% \int_\Omega (\tn K \nabla u_n) \cdot \nabla v \, \mathrm{d}x + \int_{\partial \Omega} \sigma u_n v \, \mathrm{d}s = \int_\Omega f v \, \mathrm{d}x + \int_{\partial \Omega} \sigma u_R v \, \mathrm{d}s \quad \text{pro všechna } v \in V_n.
% \]
% 
% Tato soustava je pak základem pro sestavení konečněprvkové aproximace řešení naší úlohy.
% 
% \subsection{Výpočetní síť}
% Oblast $\Omega \subset \Real^2$ předpokládáme polygonální 
% (hranice tvořena rovnými úseky). Výpočetní síť je rozdělení oblasti trojúhelníky {\it elementy} $T_e$ 
% tak, aby dva trojúhelníky sousedili buďto celou společnou hranou nebo společným vrcholem. 
% Vrcholy trojúhelníků nazýváme {\it uzly} sítě $N_j$, $j=1, \dots, n$.
% 
% Ve 2D je možné pracovat také s rozdělením na čtyřúhelníky, nebo kombinací trojúhelníků a čtyřúhelníků.
% 
% Ve 3D se používá dělení na tetraedry (čtyřstěny), deformované krychle, ale též se používají prvky 
% s různými kombinacemi trojúhelníkových a čtyřúhelníkových stěn, např. trojboké hranoly nebo pyramidy.
% 
% \subsection{Konečné prvky prvního řádu}
% V metodě konečných prvků volíme specifický podprostor $V_n$ a specifické bázové funkce v Galerkinově 
% metodě.
% 
% Volíme jako aproximační prostor $V_n$ prostor funkcí, které jsou spojité na celé oblasti $\Omega$
% a na každém elementu $T_e$ jsou polynomy prvního stupně, tedy lineární funkce. 
% Tento prostor se označuje jako prostor po částech lineárních funkcí $P^1$.
% 
% Bázové funkce $\phi_j$ jsou definovány tak, že pro každý uzel $N_j$ platí:
% 
% \[
% \phi_j(N_i) = \delta_{ij},
% \]
% 
% kde $\delta_{ij}$ je Kroneckerovo delta. To znamená, že bázová funkce $\phi_j$ nabývá hodnoty 
% 1 v uzlu $N_j$ a hodnoty 0 ve všech ostatních uzlech. Díky této vlastnosti jsou bázové funkce 
% ortogonální vůči uzlům sítě, což umožňuje snadné sestavení a řešení soustavy rovnic.
% 
% Na každém trojúhelníkovém elementu jsou bázové funkce $\phi_j$ lineární funkce definované na 
% vrcholech elementu, přičemž jejich hodnota je určena výše uvedenou podmínkou.
% 
% \subsubsection{Assemblace lineárního problému}
% Integrály vystupující ve vyhodnocení prvků matice  $\tn A$ a vektoru $\vc b$ v Galerkinově aproximaci
% si rozdělíme na integrály přes jednotlivé elementy.
% 
% Pro rovnici \eqref{eq:weak_poisson} rozdělíme integrály přes celou oblast $\Omega$ na součet integrálů přes jednotlivé elementy $T_e$:
% 
% \[
% A(u, v) = \sum_{e} \left( \int_{T_e} (\tn K \nabla u) \cdot \nabla v \, \mathrm{d}x \right) + \sum_{e} \left( \int_{\partial T_e \cap \partial \Omega} \sigma u v \, \mathrm{d}s \right).
% \]
% 
% Analogicky rozdělíme lineární funkcionál:
% 
% \[
% l(v) = \sum_{e} \left( \int_{T_e} f v \, \mathrm{d}x \right) + \sum_{e} \left( \int_{\partial T_e \cap \partial \Omega} \sigma u_R v \, \mathrm{d}s \right).
% \]
% 
% Koncept lokálních matic $\tn A_e$ a vektorů pravé strany $\vc b_e$ spočívá v tom, že pro každý element $T_e$ vypočítáme matici $\tn A_e$ s prvky:
% 
% \[
% A_e^{ij} = \int_{T_e} (\tn K \nabla \phi_j) \cdot \nabla \phi_i \, \mathrm{d}x + \int_{\partial T_e \cap \partial \Omega} \sigma \phi_j \phi_i \, \mathrm{d}s,
% \]
% 
% a vektor $\vc b_e$ s prvky:
% 
% \[
% b_e^i = \int_{T_e} f \phi_i \, \mathrm{d}x + \int_{\partial T_e \cap \partial \Omega} \sigma u_R \phi_i \, \mathrm{d}s.
% \]
% 
% Tyto lokální matice a vektory poté sestavíme do globální matice $\tn A$ a vektoru $\vc b$ podle toho, jak jsou bázové funkce $\phi_i$ a $\phi_j$ přiřazeny k uzlům.
% 
% Pro demonstraci si představme dva sousedící elementy $T_1$ a $T_2$, které sdílejí společnou hranu. Jejich lokální matice $\tn A_1$ a $\tn A_2$ přispívají k celkové globální matici $\tn A$ tak, že prvky odpovídající společným uzlům se sčítají.
% 
% Například pro uzly $i$ a $j$ sdílené oběma elementy bude prvek globální matice:
% 
% \[
% A^{ij} = A_1^{ij} + A_2^{ij}.
% \]
% 
% Tímto způsobem získáme konečnou soustavu rovnic, kde globální matice $\tn A$ je součtem všech lokálních příspěvků z jednotlivých elementů.
% 
% \subsubsection{Referenční element}
% Pro vyhodnocení lokálních matic $\tn A_e$ a vektorů $\vc b_e$ potřebujeme umět definované bázové funkce 
% vyhodnotit. Vyhodnocovat integrály přes jednotlivé elementy přímo je však nepraktické, proto 
% každý element $T_e$ zobrazíme pomocí {\it afinní transformace} na {\it referenční element},
% na kterém definujeme bázové funkce a jejich gradienty.
% 
% Afinní transformace je lineární zobrazení, které mapuje referenční element $\hat{T}$ na libovolný skutečný element $T_e$ v síti. Označíme-li souřadnice v referenčním elementu symbolem $\hat{x} = (\hat{x}_1, \hat{x}_2)$ a souřadnice ve skutečném elementu $x = (x_1, x_2)$, pak afinní transformace má tvar:
% 
% \[
% x = \tn F_e(\hat{x}) = \tn J_e \hat{x} + \vc b_e,
% \]
% 
% kde $\tn J_e$ je Jacobiho matice transformace (konstanta na každém elementu) a $\vc b_e$ je vektor posunu.
% 
% Referenční trojúhelník $\hat{T}$ definujeme jako trojúhelník s vrcholy:
% 
% \[
% \hat{N}_1 = (0, 0), \quad \hat{N}_2 = (1, 0), \quad \hat{N}_3 = (0, 1).
% \]
% 
% Na referenčním elementu definujeme bázové funkce $\hat{\phi}_i(\hat{x})$ pro $i = 1, 2, 3$ následovně:
% 
% \[
% \hat{\phi}_1(\hat{x}) = 1 - \hat{x}_1 - \hat{x}_2,
% \]
% \[
% \hat{\phi}_2(\hat{x}) = \hat{x}_1,
% \]
% \[
% \hat{\phi}_3(\hat{x}) = \hat{x}_2.
% \]
% 
% Tyto funkce jsou lineární na referenčním trojúhelníku a splňují podmínku $\hat{\phi}_i(\hat{N}_j) = \delta_{ij}$. Pomocí afinní transformace můžeme tyto bázové funkce převést na skutečný element $T_e$.
% 
% \subsubsection{Substituce forem}
% Převod integrací na referenční element.
% 
% \subsubsection{Numerická integrace}
