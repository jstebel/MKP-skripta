% \section{Diskrétní prostory konečných prvků}
% (Viz. Johnson \cite{Johnson}, kapitola 3)
% 
% V předchozích kapitolách jsme odvodili aproximaci abstraktní eliptické pomocí Galerkinovy metody.
% Vycházeli jsme přitom z předpokladu, že nekonečný prostor řešení $V$ (např. $H^1(\Omega)$) aproximujeme
% konečným prostorem $V_h$ aniž bychom přesněji specifikovali jak se tento prostor konkrétně sestrojí.
% Volbu tohoto prostoru je nutno provést tak, aby bylo možno pro jeho funkce vyhodnotit formy $a(\cdot, \cdot)$
% a $l(\cdot)$, což je nutné pro sestavení lineárního systému. Pro vyhodnocení forem je typicky potřeba 
% počítat integrály přes $\Omega$ z funkcí v prostoru $V_h$ a z jejich mocnin a derivací. 
% Základní myšlenky metody konečných prvků jsou:
% \begin{enumerate}
%  \item Rozdělit (složitou) oblast $\Omega$ na jednodušší podoblasti $K\in \mathcal T$ a vyhodnocovat integrál přes 
%  $\Omega$ jako součet integrálů přes podoblasti.
%  \item Na každé podoblasti $K$ sestrojit prostor $V_h(K)$ z jednoduchých funkcí typicky z polynomů. Polynomy zůstávají polynomy 
%  při umocňování i derivování a na jednoduchých oblastech je lze integrovat přesně.
%  \item Sestrojit $V_h(\Omega)$ jako prostor po částech polynomiálních funkcí, zajistit spojitost (nebo i spojitost derivací) mezi podoblastmi 
%  pomocí {\it stupňů volnosti}.
% \end{enumerate}
% 
% \subsection{Kostrukce konečně prvkového prostoru}
% 
% \subsection{Příklad lineárních prvků ve 2d}
% 
% \subsection{Příklad kvadratických prvků ve 2d}
% \subsection{Obecný konečný prvek}
% Předchozí příklady můžeme shrnout do obecné definice konečného prvku.
% \begin{df}
% Konečný prvek je trojice $(K, P_K, \Sigma)$, kde
% \begin{enumerate}
%  \item $K$ je konvexní podmnožina $\Real^n$ (úsečka, trojúhelních, čtyřúhelník, čtyřstěn, osmistěn, pyramida, hranol)
%  \item $P_K$ je prostor funkcí na množině $K$ (většinou polynomiální funkce, skalární nebo i vektorové). 
%  \item $\Sigma$ je množina lineárních forem na $P_K$ (o velikosti $\dim(P_K)$) splňujících podmínku unisolventnosti:
%  \[
%     \text{Pokud je $p$ funkce z $P_K$ a $\Phi( p )=0$ pro každou formu $\Phi\in\Sigma$, pak je $p\equiv0$.}
%  \]
% \end{enumerate}
% \end{df}
% 
% 
% \subsection{Vyhodnocení forem}
% Demonstrujme vyhodnocení integrálů ve formách $a(\cdot, \cdot)$, $l(\cdot)$ na následujícím integrálu, který vznikne slabou formulací Laplaceovy rovnice:
% \[
%  a(\phi_i, \phi_j) = \int_\Omega \grad \phi_i(\vc x) \cdot \grad \phi_j(\vc x) \d \vc x
% \]
% Po rozdělení integrace na elementy $K$ z triangulace $\mathcal T$:
% \[
%  a(\phi_i, \phi_j) =\sum_{K\in \mathcal T} \int_K  \grad \phi_i(\vc x) \cdot \grad \phi_j(\vc x) \d \vc x 
% \]
% Každý prvek sítě $K$ převedeme na referenční element $\hat K$ pomocí substituce 
% ${\vc x} = \tn P_K \hat{\vc x} + \vc q_K$, $\d \vc x = \abs{\tn P_K} \d \hat{ \vc x}$. 
% Při této transformaci se zachovávají hodnoty funkcí, t.j. 
% \[
%     \phi(\vc x) = \phi\big( \vc x(\hat{\vc x}) \big)=\hat \phi(\hat{\vc x}) 
% \]
% ale mění se hodnoty derivací:
% \[
%    \frac{\prtl \hat\phi(\hat{\vc x}) }{\prtl \hat{x}_i} = \frac{\prtl \phi(\vc x(\hat{\vc x})) }{\prtl \hat{x}_i} = \frac{\prtl \phi(\vc x)}{\prtl x_j}\frac{\prtl x_j}{\prtl \hat x_i}
% \]
% neboli pro gradient $\grad_{\hat{\vc x}} \hat \phi = \tn P_K \grad_{\vc x} \phi$. Po substituci dostaneme tedy integrál:
% 
% \[
%   a(\phi_i, \phi_j) = \sum_{K\in \mathcal T} \int_{\hat K} 
%         \big[ \tn P_K^{-1} \grad_{\hat{\vc x}} \hat\phi_i \big] \cdot \big[ \tn P_K^{-1} \grad_{\hat{ \vc x}} {\hat \phi}_j \big] \abs{\tn P_K} \d \hat{\vc x}.
% \]
% Pro transformaci složitějších než simplexových 
% elementů je transformace nelineární a tudíž matice $\tn P_K$ a vektor posunutí $\vc q_K$ jsou obecně závislé na $\vc x$ resp. $\hat{\vc x}$.
% Použitím numerické integrace (viz. dále) na náš integrál pak dostaneme:
% \[
%  a(\phi_i, \phi_j) = \sum_{K\in \mathcal T} \sum_{q\in Q} \vc D_i(\vc q) \cdot  \vc D_j(\vc q) J(\vc q)w_{\vc q},
% \]                   
% kde $\vc D_i(\vc q) = \tn P_K^{-1}(\vc q) \grad \hat \phi_i(\vc q)$.
% 
% \subsection{Numerická integrace}
% Pro vyhodnocení integálů je výhodné použít numerickou integraci (kvadraturu). Pro polynomiální bázové funkce lze zvolit takovou kvadraturu, že jsou 
% příslušné integrály spočteny přesně. Pokud jsou pod integrálem i jiné funkce, například vlivem koeficientů v rovnici nebo kvůli 
% transformaci na referenční element poskytuje numerická kvadratura obvykle dobrou aproximaci. Konkrétní kvadratura na referenčním prvku $\hat K$ je dána
% \begin{enumerate}
%  \item množinou kvadraturních bodů $Q=\{\vc q \in \hat K\}$,
%  \item množinou jim příslušejících vah $w_{\vc q}$.
% \end{enumerate}
% Integrál z funkce $f(\hat{\vc x})$ je pak aproximován pomocí sumy:
% \[
%     \int_{\hat K} f(\hat{\vc x}) \d \hat{\vc x} \approx \sum_{\vc q\in Q} f(\hat{\vc q}) w_{\vc q}.
% \]
% 
% Pro integraci na reálném intervalu $K\subset \Real$ lze použít {\it Gaussovu kvadraturu},  která má optimální řád chyby. Při použití $n$ kvadraturních bodů integruje
% kvadrature přesně polynomy do řádu $2n-1$. Pro vyhodnocení bilineární formy pro Laplaceovu rovnici a při použití prvků s polynomi řádu $k$, mají derivace řád $k-1$ a 
% jejich součin řád $2(k-1)$ pro přesné vyhodnocení integrálů je tedy potřeba použít $k$-bodovou Gaussovu kvadraturu. Pro 2d a 3d prvky je situace komplikovanější,
% nicméně vhodné kvadratury existují pro všechny základní typy elementů.



\section{Interpolační vlastnosti konečných prvků}
{\it Podle Johnson \cite{Johnson}, kapitola 4, bez důkazů.}

Ceovo lemma charakterizuje chybu numerického řešení pomocí 
aproximačních vlastností prostoru přibližných řešení $V_h$. 
Pro chybu přibližného řešení 
$u_h \in V_h \subset\subset V \subset\subset H^1(\Omega)$:
máme odhad:
$$
\|u - u_h\|_{H^1(\Omega)} 
\le C\inf_{v\in V_h}\|u-v\|_{H^1(\Omega)} 
\le C \|u - \pi u\|_{H^1(\Omega)}.
$$
Zde $\pi u$ značí projekci funkce $u$ z $V$ na konečně 
rozměrný aproximativní podprostor $V_h$. Konkrétně pro 
$P^1$ konečné prvky jsou koeficienty bázových funkcí dány 
hodnotami $u$ v uzlech:
$$
 \pi u(\vc x) = \sum_i=1^N u(\vc N_i) \phi_i(\vc x)
$$

Chyba aproximace již nezávisí na řešeném problému,
ale pouze na aproximované funkci, tedy samotném řešení. 
Cílem kapitoly je charakterizovat tuto aprocximační chybu.


\subsection{Seminormy $H^p$}
Koncept Hilbertových prostorů $H^1(\Omega)$ s normou 
zahrnující první derivace můžeme zobecnit na více derivací. 
Složitější parciální derivace podle více proměnných budeme 
zapisovat pomocí vektoru $\alpha = (i,j,k)$ 
počtu derivací v jednotlivých směrech
$$
  D^\alpha u = \prtl_x^i \prtl_y^j \prtl_z^k u. 
$$
Velikost vektoru $|\alpha| = i + j + k$ nám pomůže definovat
normu obecného prostoru $H^p(\Omega)$:
$$
\norm{u}_{H^p(\Omega)} = \sum_{|\alpha| \le p} 
\int_\Omega |D^\alpha u(x)|^2 \d x.
$$
To zahrnuje i vektor $\alpha=(0,0,0)$ odpovídající prosté 
$L^2$ normě, kterou můžeme také zapsat jako $H^0$:
$$
  \norm{u}_{H^0(\Omega)} = 
  \int_\Omega |D^{(0,0,0)} u(x)|^2 \d x =
  \int_\Omega |u(x)|^2 \d x.
$$
Pokud zahrneme do součtu derivace řádu právě $p$, dostaneme 
takzavanou seminormu:
$$
|u|_{H^p(\Omega)} = \sum_{|\alpha| = p} 
\int_\Omega |D^\alpha u(x)|^2 \d x.
$$

% 
% \subsection{}
%  Normu $H^1$ nejprve odhadneme seminormou, 
% tedy $L^2$ normou prvních derivací. To plyne z Poincarého nerovnosti 
% a okrajových podmínek stejně jako při důkazu elipticity. Následně normu přes 
% oblast můžme rozepsat na jednotlivé elementy:
% $$
% \|u - \pi_h[u]\|^2_{H^1(\Omega)} \le |u - w_h|^2_{H^1(\Omega)} 
% \le \sum_{K\in \mathcal{T}} \sum_{|\alpha| = 1} \Big[
% \int_K |D^\alpha u - D^\alpha w_h|^2 \Big]
% $$
% V posledním členu používáme zápis pro různé parciální derivace stejného řádu. 
% Obecně va 3D můžeme funkci $u$ derivovat podle tří různých proměnných, 
% pro složitější derivace použijeme notaci:
% $$
% \prtl_x^i \prtl_y^j \prtl_z^k u = D^\alpha u,
% $$
% kde $\alpha=(i,j,k)$ je vektor počtu derivací v jednotlivých směrech. 
% Tomuto vektoru říkáme multiindex, což je obecně vektor indexů. Pro multiindex 
% defunujeme jeho absolutní hodnotu jako součet indexů: 
% $|\alpha| = i+j+k$
% Např.
% $$
%   \prtl_x^2\prtl_z u = D^{(2, 0, 1)}
% $$
% a velikost $|(2, 0, 1)| = 3$. Tedy jde celkově o derivaci třetího řádu.
% 

\subsection{Interpolační odhady v 1D}
Odvození odhadů aproximační chyby je pro vyšší dimenze dosti komplikované, 
pro představení základních ideí si proto ukážeme odvození jednoduchého 
odhadu pouze pro 1D případ. Kvadrát normy si rozepíšeme na jednotlivé 
elementy:

$$
\|u - \pi_u]\|^2_{H^1(\Omega)} = 
\sum_{K\in \mathcal{T}}  \|u - \pi u\|^2_{H^1(K)}
$$


Pro jednoduchost značení uvažujeme obecný element posunutí do počátku 
$x_i = 0$, $x_{i+1}=h$ s příslušně posunutými funkcemi. 
Kvadrát chyby na elementu pak zahrnuje $L^2$ normu chyby a $L^2$ 
normu její derivace:
\def\intK{\int_{0}^{h}}
\begin{equation}
\label{eq:h1_approx_err}
\|u - \pi u\|^2_{H^1(K)} = \intK |u - \pi u|^2 \d x 
+ \intK |(u - \pi u)'|^2 \d x
\end{equation}
Absolutní hodnoty $e(x) = |u-\pi u|(x)$ uvnitř druhých mocnin nejsou  pro skalární řešení nutné, 
ale byly by nutné pro řešení s komplexními nebo vektorovými hodnotami.

\subsubsection{$L^2$ norma}
Jelikož je chyba $e(x)$ nulová v krajních bodech intervalu, je hodnota chyby v prvním 
integrálu v \eqref{eq:h1_approx_err} určena hodnootu derivace a můžeme ji 
odhadnout pomocí druhého integrálu. Jelikož $e(0) = 0$, můžeme
napsat $e^2(x)$ jako integrál z derivace (základní věta kalkulu):
\begin{equation}
    \label{eq:tylor_0}
    \intK e^2(x) \d x 
    = \intK e(0) + \int_{0}^x (e^2(t))' \d t \d x
    =\intK \int_{0}^x 2e(t) e'(t)\d t  \d x
    \le h \intK 2|e(t) e'(t)|\d t 
\end{equation}

Pro poslední nerovnost jsme nejprve aplikovali absolutní hodnotu a následně prodloužili integraci 
z $(0,x)$ na $(0,h)$. Dále použijeme AG nerovnost:
\begin{equation}
  \label{eq:ag_ineq}
  2ab \le a^2+b^2,
\end{equation}
která plyne z $0\le(a-b)^2$. Dosazením $a= e(t)/\sqrt{2h}$ a $b = \sqrt{2h}e'(t)$ 
dostaneme:
$$
h \intK 2|e(t) e'(t)|\d t 
\le h\frac{1}{2h}\int_0^h |e(t)|^2\d t + h(2h)\int_0^h |e'(t)|^2\d t
= \frac12\int_0^h |e(t)|^2\d t + 2h^2\int_0^h |e'(t)|^2\d t
$$
Nyní už jen odečteme první integrál od obou stran a vynásobíme dvěma:
$$
\int_0^h e^2(x) \d x \le 4h^2 \int_0^h |e'(x)|^2 \d x 
$$



% f'(a) + \int_a^b f'' = f'(b) 
\subsubsection{$H^1$ seminorma}
Dále se pustíme do odhadu $H^1$ seminormy, tedy druhého integrálu v \eqref{eq:h1_approx_err}.
Celý postup je založen na stejných ingrediencích jako pro $L^2$, ale v rafinovanější 
podobě. Opakovaným použitím základní věty kalkulu odvodíme approximaci Taylorovým 
polynomem s integrálním tvarem zbytku. Po dosazení tento zbytek odpovídá 
odhadované chybě. 

\begin{theorem}[Taylor 2. řádu]
Pro funkci $u:\Real -> \Real$, která má integrovatelnou druhou derivaci na intervalu 
$I$ 
($u''\in L^1(I)$ platí pro $a,b \in I$:

$$
u(b) = u(a) + (b-a)u'(a) + \int_a^b u''(s) (b-s) \d s.
$$
\end{theorem}
\begin{proof}
Začneme základní větou kalkulu:
$$
u(b) = u(a) + \int_a^b u'(t) \d t
$$
%
Aplikováním stejného vztahu pro $u'(t)$ máme:
$$
u'(t) = u'(a) +\int_a^t u''(s) \d s
$$
Dosazením do prvního vztahu a změnou zápisu oblasti dvojitého integrálu
získáme aproximaci Taylorovým polynomem druhého řádu:
\begin{align*}
u(b) &= u(a) + \int_a^b u'(a) \d t + \int_a^b \int_a^t u''(s) \d s \d t\\
     &= u(a) + (b-a)u'(a) + \int_a^b \int_s^b u''(s) \d t \d s\\
     &= u(a) + (b-a)u'(a) + \int_a^b u''(s) (b-s) \d s.
\end{align*}
Druhá rovnost je trochu záludná, na levé i pravé straně je 
dvojitý integrál přes všechny dvojice $(s, t)$ splňující $a\le s\le  t \le b$.
Tyto dvojice tvoří troúhleník, přes který integrujeme buďto přes horizontální 
úsečky $s\in(a, t)$ (levá strana) nebo přes vertikální úsečky $t\in (s, b)$ 
(pravá strana).
\end{proof}

Taylora použijeme pro $b=h$, $a=x$:
$$
u(h) = u(x) + (h-x)u'(x) + \int_x^{h} u''(s) (h-s) \d s
$$
a pro $b=0$, $a=x$:
$$
u(0) = u(x) + (0-x)u'(x) + \int_x^{0} u''(s) (0-s) \d s
$$
%
Odečtením a vydělíme $h$ dostaneme derivaci $P^1$ approximace:
$$
(\pi u)' = \frac{u(h) - u(0)}{h} =  u'(x) + \frac{1}{h}I_1(x) - \frac{1}{h}I_2(x)
$$
a zapíšeme chybu mezi derivací přesného a přibližného řešení:
$$
|(u-\pi u)'(x)| = \frac{1}{h}\Big|I_1(x) - I_2(x)\Big|
$$
%
Nyní umocníme a integrujeme:
$$
\int_{0}^{h} |(u-\pi u)'|^2 \d x
= \int_{0}^{h} |(I_1(x) - I_2(x))|^2\d x
\le \frac{1}{h}\int_{0}^{h} |I_1(x)|^2 + |I_2(x)|^2 \d x
$$
Kde jsme použili trojúhelníkovou nerovnost: $|a+b|^2 \le |a|^2 + |b|^2$.
%
Dále oba integrály odhadneme:
$$
|I_1| \le \int_x^h |u''(s)| |h-s| \d s \le h \int_0^h |u''(s)| \d s
$$
a
$$
|I_2| \le \int_0^x |u''(s)| |0-s| \d s \le h \int_0^h |u''(s)| \d s.
$$
Po dosazení dostaneme odhad:
\begin{equation}
\label{eq:error_est}
\int_{0}^{h} |(u-\pi u)'|^2 \d x
\le 2\int_{0}^{h} \Big|\int_0^h|u''(s)|\d s\Big|^2\d x
\le h \left|\int_0^h|u''(s)|\d s\right|^2
\end{equation}
Pro dokončení potřebujeme Cauchy-Schwartzovu nerovnost, 
která dostane mocninu dovnitř integrálu. V jednoduché podobě ji znadno odvodíme z 
AG nerovnosti $xy\le \frac12(x^2+y^2)$. 
Dosadíme $x=f(s)$ $y=f(t)$ a integrujeme přes $s$ a $t$:
$$
\left(\int_a^b f(s)\d s\right)^2 = \int_a^b\int_a^b f(s)f(t)\d s\d t 
\le \frac12\int_a^b\int_a^bf^2(s) +  f^2(t)\d s \d t
=  (b-a)\int_a^b f^2(s) \d s
$$
Aplikací na pravou stranu \eqref{eq:error_est} získáme finální odhad:
$$
\int_{0}^{h} |(u-\pi u)'|^2 \d x
\le h^2 \int_0^h|u''(s)|^2\d s
$$

\subsection{Aproximační odhady obecné}
V předchozí sekci jsme odhadli chybu aproximace pro lineární konečné prvky v
1D. Aplikací podobných postupů lze odvodit odhady pro obecný řád prvků $P^p$ 
a dimenzi $d$. Pro 
$$
  \|u-\pi u\|_{H^1(K)}\le h^p|u|_{H^{r+2}}(K)
$$
$$
  \|u-\pi u\|_{L^2(K)}\le h^{p+1}|u|_{H^{r+2}}(K)
$$

% 
% \subsection{Regulární síť}
% 
% Nechť $\mathcal{T}_h$ je triangulace, rozdělení oblasti $\Omega$ na konečné prvky, např.  
% trojúhleníky (2D) nebo tetrahedrony (3D). Pro prvek $K \in \mathcal{T}_h$ označíme:
% \begin{align*}
% &h_K \text{ délku nejdelší hrany elementu $K$}\\
% &\rho_K \text{ poloměr vepsané kružnice/koule elementu $K$}\\
% &h = max_{K\in \mathcal{T}_h} h_K
% \end{align*}
% Pro jednoduchost uvažujeme přibližně stejné hodnoty $h_K$ napříč elementy. 
% Pokud bychom při pevné velikosti $h_K$ měli elementy s velmi malým $\rho_K$, velmi 'tenké' trojúhelníky
% s vrcholy skoro na jedné přímce, příslušné bázové funkce si jsou stále více blízké a lokální matice
% se blíží singulární matici. Uvažujme proto takzavnou regulární síť, kde existuje konstanta $\beta$:
% $$
% \frac{\rho_K}{h_K} \ge \beta.
% $$
% 
% \subsection{Interpolační odhady ve 2D a 3D}
% Prvně uvažujme $P^1$ konečné prvky, $V=H^1(\Omega)$, $V_h=\{v\in V: v|K \in P^1(K)\}$.
% Projekci z $V$ na $V_h$ oznažíme $\pi$. Pak platí
% \begin{theorem}
% $$
% \|v-\pi v\|_{L_2(K)} \le C h^2_K \|v\|_{H^2(K)}
% $$
% a
% $$
% \|v-\pi v\|_{H^1(K)} \le C h^2_K \|v\|_{H^2(K)}
% $$
% 
% \end{theorem}
% 
% 
% 
% 
% Obecně platí následující odhady:
% 
% 
% \begin{itemize}
%  \item Interpolační vlastnosti lineárních prvků ve 2D.
%  \item Interpolační vlastnosti prvků vyšších řádů.
%  \item Aplikace interpolačních odhadů v Ceově lemmatu.
%  \item Regularita přesného slabého řešení (Kdy je možno použít prvky vyšších řádů?)
%  \item Adaptivní metody, základní principy.
% \end{itemize}

% 






