\section{Diskrétní prostory konečných prvků}
(Viz. Johnson \cite{Johnson}, kapitola 3)

V předchozích kapitolách jsme odvodili aproximaci abstraktní eliptické pomocí Galerkinovy metody.
Vycházeli jsme přitom z předpokladu, že nekonečný prostor řešení $V$ (např. $H^1(\Omega)$) aproximujeme
konečným prostorem $V_h$ aniž bychom přesněji specifikovali jak se tento prostor konkrétně sestrojí.
Volbu tohoto prostoru je nutno provést tak, aby bylo možno pro jeho funkce vyhodnotit formy $a(\cdot, \cdot)$
a $l(\cdot)$, což je nutné pro sestavení lineárního systému. Pro vyhodnocení forem je typicky potřeba 
počítat integrály přes $\Omega$ z funkcí v prostoru $V_h$ a z jejich mocnin a derivací. 
Základní myšlenky metody konečných prvků jsou:
\begin{enumerate}
 \item Rozdělit (složitou) oblast $\Omega$ na jednodušší podoblasti $K\in \mathcal T$ a vyhodnocovat integrál přes 
 $\Omega$ jako součet integrálů přes podoblasti.
 \item Na každé podoblasti $K$ sestrojit prostor $V_h(K)$ z jednoduchých funkcí typicky z polynomů. Polynomy zůstávají polynomy 
 při umocňování i derivování a na jednoduchých oblastech je lze integrovat přesně.
 \item Sestrojit $V_h(\Omega)$ jako prostor po částech polynomiálních funkcí, zajistit spojitost (nebo i spojitost derivací) mezi podoblastmi 
 pomocí {\it stupňů volnosti}.
\end{enumerate}

\subsection{Kostrukce konečně prvkového prostoru}

\subsection{Příklad lineárních prvků ve 2d}

\subsection{Příklad kvadratických prvků ve 2d}
\subsection{Obecný konečný prvek}
Předchozí příklady můžeme shrnout do obecné definice konečného prvku.
\begin{df}
Konečný prvek je trojice $(K, P_K, \Sigma)$, kde
\begin{enumerate}
 \item $K$ je konvexní podmnožina $\Real^n$ (úsečka, trojúhelních, čtyřúhelník, čtyřstěn, osmistěn, pyramida, hranol)
 \item $P_K$ je prostor funkcí na množině $K$ (většinou polynomiální funkce, skalární nebo i vektorové). 
 \item $\Sigma$ je množina lineárních forem na $P_K$ (o velikosti $\dim(P_K)$) splňujících podmínku unisolventnosti:
 \[
    \text{Pokud je $p$ funkce z $P_K$ a $\Phi( p )=0$ pro každou formu $\Phi\in\Sigma$, pak je $p\equiv0$.}
 \]
\end{enumerate}
\end{df}


\subsection{Vyhodnocení forem}
Demonstrujme vyhodnocení integrálů ve formách $a(\cdot, \cdot)$, $l(\cdot)$ na následujícím integrálu, který vznikne slabou formulací Laplaceovy rovnice:
\[
 a(\phi_i, \phi_j) = \int_\Omega \grad \phi_i(\vc x) \cdot \grad \phi_j(\vc x) \d \vc x
\]
Po rozdělení integrace na elementy $K$ z triangulace $\mathcal T$:
\[
 a(\phi_i, \phi_j) =\sum_{K\in \mathcal T} \int_K  \grad \phi_i(\vc x) \cdot \grad \phi_j(\vc x) \d \vc x 
\]
Každý prvek sítě $K$ převedeme na referenční element $\hat K$ pomocí substituce 
${\vc x} = \tn P_K \hat{\vc x} + \vc q_K$, $\d \vc x = \abs{\tn P_K} \d \hat{ \vc x}$. 
Při této transformaci se zachovávají hodnoty funkcí, t.j. 
\[
    \phi(\vc x) = \phi\big( \vc x(\hat{\vc x}) \big)=\hat \phi(\hat{\vc x}) 
\]
ale mění se hodnoty derivací:
\[
   \frac{\prtl \hat\phi(\hat{\vc x}) }{\prtl \hat{x}_i} = \frac{\prtl \phi(\vc x(\hat{\vc x})) }{\prtl \hat{x}_i} = \frac{\prtl \phi(\vc x)}{\prtl x_j}\frac{\prtl x_j}{\prtl \hat x_i}
\]
neboli pro gradient $\grad_{\hat{\vc x}} \hat \phi = \tn P_K \grad_{\vc x} \phi$. Po substituci dostaneme tedy integrál:

\[
  a(\phi_i, \phi_j) = \sum_{K\in \mathcal T} \int_{\hat K} 
        \big[ \tn P_K^{-1} \grad_{\hat{\vc x}} \hat\phi_i \big] \cdot \big[ \tn P_K^{-1} \grad_{\hat{ \vc x}} {\hat \phi}_j \big] \abs{\tn P_K} \d \hat{\vc x}.
\]
Pro transformaci složitějších než simplexových 
elementů je transformace nelineární a tudíž matice $\tn P_K$ a vektor posunutí $\vc q_K$ jsou obecně závislé na $\vc x$ resp. $\hat{\vc x}$.
Použitím numerické integrace (viz. dále) na náš integrál pak dostaneme:
\[
 a(\phi_i, \phi_j) = \sum_{K\in \mathcal T} \sum_{q\in Q} \vc D_i(\vc q) \cdot  \vc D_j(\vc q) J(\vc q)w_{\vc q},
\]                   
kde $\vc D_i(\vc q) = \tn P_K^{-1}(\vc q) \grad \hat \phi_i(\vc q)$.

\subsection{Numerická integrace}
Pro vyhodnocení integálů je výhodné použít numerickou integraci (kvadraturu). Pro polynomiální bázové funkce lze zvolit takovou kvadraturu, že jsou 
příslušné integrály spočteny přesně. Pokud jsou pod integrálem i jiné funkce, například vlivem koeficientů v rovnici nebo kvůli 
transformaci na referenční element poskytuje numerická kvadratura obvykle dobrou aproximaci. Konkrétní kvadratura na referenčním prvku $\hat K$ je dána
\begin{enumerate}
 \item množinou kvadraturních bodů $Q=\{\vc q \in \hat K\}$,
 \item množinou jim příslušejících vah $w_{\vc q}$.
\end{enumerate}
Integrál z funkce $f(\hat{\vc x})$ je pak aproximován pomocí sumy:
\[
    \int_{\hat K} f(\hat{\vc x}) \d \hat{\vc x} \approx \sum_{\vc q\in Q} f(\hat{\vc q}) w_{\vc q}.
\]

Pro integraci na reálném intervalu $K\subset \Real$ lze použít {\it Gaussovu kvadraturu},  která má optimální řád chyby. Při použití $n$ kvadraturních bodů integruje
kvadrature přesně polynomy do řádu $2n-1$. Pro vyhodnocení bilineární formy pro Laplaceovu rovnici a při použití prvků s polynomi řádu $k$, mají derivace řád $k-1$ a 
jejich součin řád $2(k-1)$ pro přesné vyhodnocení integrálů je tedy potřeba použít $k$-bodovou Gaussovu kvadraturu. Pro 2d a 3d prvky je situace komplikovanější,
nicméně vhodné kvadratury existují pro všechny základní typy elementů.

\section{Interpolační vlastnosti konečných prvků}
Podle Johnson \cite{Johnson}, kapitola 4, bez důkazů.
\begin{itemize}
 \item Interpolační vlastnosti lineárních prvků ve 2D.
 \item Interpolační vlastnosti prvků vyšších řádů.
 \item Aplikace interpolačních odhadů v Ceově lemmatu.
 \item Regularita přesného slabého řešení (Kdy je možno použít prvky vyšších řádů?)
 \item Adaptivní metody, základní principy.
\end{itemize}

% 






