\section{Slabá formulace pro eliptické úlohy}

\subsection{Slabé řešení rovnice}
\subsubsection{Intuitivní motivace}
\begin{itemize}
 \item Veličina jako pole je abstrakce, měření veličiny (např. teploty $T$) v prostoru odpovídá průměrování 
       přes okolí měřeného bodu $\vc y$ s váhovou (testovací funkcí) $v$ ve tvaru ``kopečku'' 
       na okolí bodu $\vc y$:
       \[
       T_{\vc y} = \int T(\vc x) v(\vc x - \vc y) \d \vc x
       \]
 \item Zákony zachování jsou formulovány jako bilance nějaké veličiny přes {\bf libovolnou} oblast 
       $\Omega$. Pokud použijeme charakteristickou funkci oblasti, $v(\vc x) = 1$ 
       právě když $\vc x \in \Omega$ jinak $v(\vc x) = 0$, můžeme zachování hmoty, 
       \emph{rovnici kontinuity} zapsat:
       
       \[
          \frac{\d}{\d t} \int T v \d x = \int \frac{\d}{\d t} T v 
          + T \frac{\d}{\d \vc x}v \frac{\d}{\d t} \vc x = \int \prtl_t T v + T \grad v \cdot \vc V,
       \]
       kde $\vc V$ je rychlost proudění. Pokud bychom použili ``zhlazenou'' charakteristickou funkcí 
       davají uvedené derivace smysl. 
       
 \item Rovnice bychom tedy měli spíše uvažovat v podobné ``integrální'' formulaci, 
       kde nemusíme mít k definovány bodové hodnoty veličiny $T$ a jejích derivací. 
       Intergální = slabá formulace umožní matematicky správně pracovat s nehladkými řešeními. 
\end{itemize}



\subsubsection{Další motivační úvahy}
Uvažujme transportní rovnici na nekonečné oblasti $\Omega = \Real$ (pro $x$):
\begin{equation}
    \label{eq:point_transport_1d}
    \prtl_t u(t,x) + v\prtl_x u(t,x) =0 
\end{equation}
kde $v$ je konstantní rychlost. Řešením je posunutá počáteční podmínka $u_0(x)$:
\[
    u(t,x) = u_0(x-vt)
\]
pro libovolnou diferencovatelnou funkci $u_0$ máme:
\[
    \prtl_t u + v\prtl_x u=u_0'(x-vt)(-v) + vu_0'(x-vt)=0
\]

Zdá se logické, že by toto ``posouvání řešení'' mělo platit pro libovolnou počáteční podmínku 
$u_0$, ale rovnice je formulována tak, že to platí jen pokud je $u_0$ diferencovatelná. 
Nešlo by s tou formulací něco udělat?

Zkusme se vrátit k tomu jak jsme transportní rovnici odvodili. Pomocí Raynoldsovy věty jsme dostali:
\[
    \int_{\Omega_t} \prtl_t u + \div(v u) \d x= \int_{\Omega_t} \prtl_t u + v\prtl_x u \d x= 0
\]
pro libovolnou oblast $\Omega_t$. A bodovou rovnici \eqref{eq:point_transport_1d} 
jsme dostali za předpokladu, že vnitřek integrálu je spojitý, 
tedy $u$ je spojitě diferencovatelná v prostoru i čase. 
Tedy požadavek na diferencovatelnost je ve skutečnosti umělý a i bez něj platí:
\[
    \int_\Real \phi(\prtl_t u + \prtl_x(vu)) \d x = 0
\]
pro libovolnou hladkou funkci $\phi$ s kompaktním nosičem. 
Nosič funkce je množina, kde je funkce nenulová:
\[
    \supp \phi = \{ x,\ \phi(x) > 0\}
\]
a v našem případě jsou kompaktní všechny omezené a uzavřené intervaly. 
Jde tedy o to, že $\phi$ musí být \emph{směrem k nekonečnu} nulová.

Nyní však nevíme co je $\prtl_t u$ pokud je $u_0$ nespojitá, 
např. $u_0(x)=\sgn(x)$. Abychom se tohoto problému zbavili použijeme Greenovu větu 
(zde vlastně jen integraci per partes):

\[
    \int_\Real \int_\Real \phi(t,x)\big(\prtl_t u + \prtl_x(vu)\big) \d x \d t= 
    \int_\Real \int_\Real -\prtl_t \phi u - \prtl_x \phi vu \d x \d t = 0
\]
pro libovolnou hladkou funkci $\phi(t,x)$ s kompaktním nosičem na $\Real \times \Real$.
Tato rovnice již skutečně platí pro libovolnou integrovatelnou počáteční podmínku $u_0$.


\subsubsection{Klasická formulace eliptické úlohy}
\label{sec:weak_sol_elliptic_eq}
Budeme řešit rovnici
\begin{align}
   \div(\vc q) =f + \sigma_f(u_f - u) \quad\text{na }\Omega \\
   \vc q = -\tn K\grad u
    \label{eq:elliptic}
\end{align}
Rovnice popisuje ustálené rozložení teploty, dále značené $u(\vc x)$, v tělese $\Omega$. 
Pro jednoduchost zde uvažujeme pouze vedení tepla bez konvekce, kde teplený tok $\vc q$ je dán 
anisotropní teplnou vodivostí popsanou tensorem $\tn K$. 

Na pravé straně je hustota objemových zdrojů 
tepla $f$ (např. produkce tepla v čipu) a kontaktní zdroj tepla. 
Kontaktní zdroj modeluje například přenos tepla z tělesa na chladič v podobě kovového plátu.
Nedokonalý přenos tepla při dotyku chladiče je popsán koeficientem přestupu tepla $\sigma_f\ge 0$ 
z tělesa na plát, který je udržován na teplotě prostředí $u_f$ (velmi zjednodušující).


Eliptickou rovnici na omezené oblasti $\Omega$ je třeba doplnit okrajovými podmínkami 
{\bf na celé hranici} $\prtl \Omega$. Základní tři typy podmínek jsou:

{\bf Dirichletova okrajová podmínka.} 
\[
    u(\vc x) = u_d(\vc x)\text{ na }\Gamma_d
\]
předepisuje teplotu $u_d$ na části hranice $\Gamma_d$. Pevná teplota na hranici modeluje situaci, kdy se těleso $\Omega$ dokonale vodivě dotýká termostatu - 
tělesa z velkou teplenou kapacitou.

{\bf Neumannova okrajová podmínka.}
\[
    \vc q \cdot \vc n = (-\tn K \grad u) \cdot \vc n = q_n\text{ na }\Gamma_n.
\]
Člen vlevo se nazývá normálový teplený tok, který je předepsán jako $q_n$. V teorii se obvykle uvažuje jako kladný tok ve směru ven z oblasti (vnější normála), 
nicmáně z důvodu konzistence s objemovými zdroji se v praxi používá raději opačná konvence. Fyzikálně relevantní je případ kdy je hranice pevná 
$\vc v = 0$ a teplený tok je dán např. výkonem topidla na hranici. 

{\bf Robinova (Newtonova) okrajová podmínka.}
\[
     \vc q \cdot \vc n = (-\tn K \grad u)  \cdot \vc n = \sigma_r(u - u_r)\text{ na }\Gamma_r. 
\]
Opět je releventní především případ $\vc v =0$, kdy podmínka modeluje realistický přenost tepla s koeficientem $\sigma_r\ge 0$ z tělesa o teplotě $u_r$.
Zde je opět $\vc n$ vnější normála, tedy vlevo je tok ven z oblasti, který je kladný pokud je $u > u_r$ což souhlasí se znaménkem na pravé straně.

Mimo tyto základní podmínky se v reálných úlohách objevují i nejrůznější další podmínky. 
Například kombinace Neumannovy a Robinovy okrajové podmínky:
\[
    -\tn K \grad u \cdot \vc n = \sigma_r(u - u_r) + q_n
\]

Dále předpokládáme, že množiny $\Gamma_d$, $\Gamma_n$, a $\Gamma_r$ jsou navzájem diskjunktní (nemají průnik) a jejich sjednocení (respektive sjednocení jejich uzávěrů)
je hranice $\prtl \Omega$.


\subsubsection{Slabá formulace eliptické úlohy}
\label{sec:weak_form_elliptic_eq}

V této kapitole odvodíme slabou formulaci rovnice vedení tepla, 
spolu s aplikací klasických okrajových podmínek.

Prvně přenásobíme rovnici \eqref{eq:elliptic} libovolnou hladkou testovací funkcí 
$v(\vc x) \in C^\infty(\ol\Omega)$ (hladá až do hranice) a integrujeme přes $\Omega$:
\[
    \int_\Omega v\Big(-\div(\tn K\grad u\Big) \d \vc x = \int_\Omega v \Big(f + \sigma_f(u_f - u)\Big) \d \vc x.
\]
Dále v na levé straně použijeme Greenovu větu k přehození divergence na testovací funkci:
\begin{equation}
    \label{eq:after_green}
    \int_\Omega \grad v \cdot (\tn K\grad u) \d \vc x + \int_{\prtl \Omega} -v(\tn K\grad u)\cdot \vc n \d s 
    = \int_\Omega v \Big(f + \sigma_f(u_f - u)\Big) \d \vc x.
\end{equation}
Než přistoupíme k aplikaci okrajových podmínek, zamysleme se jaké vlastnosti musí mít funkce $u$, 
aby tato rovnice vůbec měla smysl. Pokud budeme předpokládat, že 
všechny ostatní parametry jsou hladké, musí mít funkce $u$ alespoň integrovatelné derivace 
($\prtl_{x_i} u \in L_1(\Omega)$). Každopádně funkce $u$ nemůže být libovolná, 
ale patří do nějakého vektorového prostoru funkcí $H^1(\Omega)$, který si přesně zavedeme až později,
ale již víme, že jeho funkce mají integrovatelnou derivaci.

Pro začátek předpokládejme pouze \emph{homogenní} Dirichletovu okrajovou podmínku $u_d =0$. 
Pak je řešení $u$ ve skutečnosti z podprostoru:
\[
    V_0 = \{ u\in H^1(\Omega), u(\vc x) = 0\text{ na } \Gamma_d \}
\]
Jelikož už známe hodnotu řešení na hranici $\Gamma_d$ nepotřebujeme řešení určit rovnicí 
\eqref{eq:after_green} a testovací funkce proto může být na této části hranice také nulová, t.j. 
testovací funkce $v$ bude z prostoru:
\[
     \mathcal D_0 = \{\phi \in C^\infty(\ol\Omega),\ v(\vc x) = 0\text{ na } \Gamma_d\}.
\]

Nyní rozdělíme hraniční integrál na integrály přes části hranice odpovídající 
jednotlivým typům okrajových podmínek:
Na $\Gamma_d$ je $v=0$ příslušná člen je tedy nulový:
\[
    \int_{\Gamma_d} -v(\tn K\grad u)\cdot \vc n \d s = 0
\]

Na $\Gamma_n$ je tok roven $q_n$, máme tedy:
\[
    \int_{\Gamma_n} -v(\tn K\grad u)\cdot \vc n \d s = \int_{\Gamma_n} v q_n \d s.
\]

Podobně známe tok na $\Gamma_r$:
\[
    \int_{\Gamma_r} -v(\tn K\grad u)\cdot \vc n \d s = \int_{\Gamma_n} v \sigma_r(u - u_r) \d s.
\]

Dostáváme tak slabou formulaci rovnice \eqref{ew:elliptic}. Slabým řešením bude každá funkce $u$ z prostoru $V_0$, 
která splňuje
\begin{align}
    \label{eq:weak_homo_dirich}
    A(u, v) &:= \int_\Omega (\tn K\grad u) \cdot \grad v   +   \sigma_f u v \d \vc x 
              + \int_{\Gamma_r} \sigma_r u v \d s \\
              &=  \int_\Omega \Big(f + \sigma_f u_f)\Big)v \d \vc x + \int_{\Gamma_r} \sigma_r u_r v \d s 
              + \int_{\Gamma_n} -q_n v \d s =: l(v)
\end{align}
pro všechna $v \in \mathcal D$. Levá strana rovnice závisí lineárně na testovací funkci $v$ i na řešení $u$ 
a souhrně ji tak můžeme označit lineární funkcí dvou proměnných $A(u, v)$, které proto nazýváme bi-lineární forma.
``bi'' protože je lineární v obou proměnných, ``forma'' má na vstupu funkce $u$ a $v$, ale výstupem je reálné číslo.
Podobně pravá strana rovnice je lineárně závislá pouze na $v$, je to tedy lineární forma $l(v)$.

Nyní uvažujme případ s obecnou Dirichletovou okrajovou podmínkou $u_d\ne 0$. 
Budeme předpokládat, že $u_d$ lze prodloužit dovnitř oblasti $\Omega$. Tedy, že existuje 
funkce $\tilde u_d$ z prostoru $H^1(\Omega)$, která se shoduje s okrajovou podmínkou na hranici,
 $\tilde u_d = u_d$ na $\Gamma_d$, a je libovolná uvnitř oblasti (tam je omezená pouze prostorem, tedy má jistou malou hladkost). 
 
 
Pak řešení s obecnou Dirichletovou podmínkou
rozložíme $u = \tilde u_d + u_0$. Kompletní řešení $u$ je na hranici nenulové, ale $u_0$ je na hranici nulové
protože $u=\tilde u_d$ na hranici. Proto je $u_0$ z prostoru $V_0$. Funkce $u_0$ představuje neznámou 
část řešení, kterou dostaneme řešením problému:
\[
    A(u_0, v) = l(v) - A(\tilde u_d, v) = \tilde l(v)
\]
Zjednodušeně, rozšířenou okrajovou podmínku dosadíme do bilineární formy a tím z ní uděláme pouze 
lineární formu a přesuneme na pravou stranu. Změna Dirichletovy podmínky tak mění pouze pravou stranu 
rovnice podobně jako ostatní okrajové podmínky.

{\bf Souvislost s kvantovou mechanikou.} Bilineární a lineární formy se používají i v kvantové mechanice 
a značí se pomocí bra-ket notace. Slabou formulaci bychom v této notaci zapsali:
\[
    \langle u | A | v \rangle = \langle l | v \rangle
\]
Bilineární formu můžeme chápat jako operátor $A$, který z řešení $u$ vyrobí lineární formu (bra) $\langle u| A$ a
ta se vynásobí skalárním součinem s testovací funkcí (ket) $|v\rangle$.


\subsection{Odvození klasické formulace ze slabé}
Doposud jsme odvodili, že ze silné formulace plyne splnění slabé formulace, ale není zřejmé, že jsme 
tímto postupem napřišli o nějakou informaci, tedy zda je ze slabé formulace plyne silná pokud 
z nějakého důvodu bude slabé řešení hladší, konkrétně se spojitými druhými derivacemi: $u\in C^2(\Omega)$.

V principu všechny výchozí rovnice musíme dostat aplikováním vhodných testovacích funkcí. 
Nejprve odvodíme rovnici uvnitř $\Omega$. Ve slabé formulaci  \eqref{eq:weak_homo_dirich} 
budeme testovat funkcí $v$ s nosičem uvnitř $\Omega$, t.j. $v$ je nulová na hranici. 
Hraniční intergrály jsou tedy nulové a použitím Greenovy věty, zpětně dostaneme:
\begin{equation}
    \label{eq:green_interior}
    \int_\Omega -\div (\tn K\grad u) v \d \vc x = \int_\Omega \Big(f + \sigma_f (u_f - u)\Big)v \d \vc x.
\end{equation}
Odtud plyne splnění bodové rovnice \eqref{eq:elliptic} v každém bodě uvnitř $\Omega$.

Dále potřebujeme odvodit splnění okrajových podmínek. Dirichletova podmínka je splněna přímo volbou prostoru, 
jelikož $u=\tilde u_d + u_0$ a $u_0$ je nulová na $\Gamma_d$.

Pro odvození Neumannovy a Robinovy okrajové podmínky uvažujeme libovolnou hladkou
 funkci $v$ s nosičem na hranici $\Gamma_{nr} =\Gamma_n \cup \Gamma_r$. 
Funkce je nulová uvnitř $\Omega$.
Dále tuto funkci hladce prodloužíme dovnitř  $\Omega$ (jako jsme to uvažovali pro Dirichletovu podmínku),
ale tak aby měla nosič pouze do vzdálenosti $\eps$ od hranice. Výslednou funkci  $v_\epsilon \in \mathcal D_0$
použijeme jako testovací v \eqref{eq:weak_homo_dirich} a aplikujeme Greenovu větu:
\[
    \int_\Omega X(u)v_\epsilon \d \vc x + \int_{\Gamma_{nr}} (\tn K\grad u-\vc v u)\cdot \vc n v_\epsilon \d s    
              =  \int_{\Gamma_r} \sigma_r (u_r -u) v_\epsilon \d s + \int_{\Gamma_n} -q v_\epsilon \d s 
\]
kde první člen obsahuje všechny členy z předchozí rovnice \eqref{eq:green_interior}. 
Nyní provedeme limitu $\epsilon \to 0$. V této limitě se první intergál 
bude blížit nule, jelikož $v_\epsilon$ je nenulová na množině velikosti $\epsilon \times \abs{\Gamma_nr}$, 
což konverguje k nule. Naproti tomu ve zbylých 
hraničních integrálech je $v_\epsilon = v$, což na $\epsilon$ nezávisí, tedy dostaneme:
\[
    \int_{\Gamma_{nr}} -(\tn K\grad u)\cdot \vc n v \d s  
              =  \int_{\Gamma_r} \sigma_r (u -u_r) v \d s + \int_{\Gamma_n} q v \d s 
\]
a odtud obě okrajové podmínky.


% JB TODO: pokud přidáme přednášku pro nestacionární úlohy, tak uvést rovnici s advekcí
% a zmínit problematiku podmínek závislých na rychlosti.
% \subsubsection{Advekčně difúzní případ}
% Například pokud na části hranice vtéká do oblasti voda o rychlosti $\vc V$ o dané teplotě $U$, 
% půjde o Neumannovu podmínku s $\vc v\ne 0$ a $q_n=U\vc v \cdot \vc n$. Pro výtok z oblasti však potřebujeme podmínku $q_n=u\vc v \cdot \vc n$, což lze považovat za Robinovu podmínku s $u_r=0$ 
% a $\sigma_r=\vc v \cdot\vc n$. Ovšem rychlost $\vc v$ nemusí být na hranici dopředu známa, může být výsledkem řešení nějaké rovnice proudění, proto 
% je obvykle tuto podmínku chýpat jako zvláštní typ.


\section{MKP pro eliptické úlohy}
\subsection{Galerkinova aproximace}
{\bf Konečněrozměrný podprostor $V_0^n$.} Řešení (a jak uvidíme ve slabé formulaci

{\bf Volba báze.}

{\bf Lineární systém.}
\[
 A\Big(\sum_{j=1}^n u_j\phi_j, \phi_i\Big) = l(\phi_i)
\]


\subsection{Konečné prvky prvního řádu}
\subsubsection{Referenční element}
\subsubsection{Bázové funkce}
\subsubsection{Substituce forem}
Převod integrací na referenční element.
\subsubsection{Numerická integrace}
\subsubsection{Assemblace lineárního problému}
