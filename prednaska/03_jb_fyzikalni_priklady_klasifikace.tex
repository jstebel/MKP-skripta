\section{Transportní procesy}
Rovnice vedení tepla odvozená v předchozí kapitole je jedním z příkladů transportních procesů, které popisují transport nějaké veličiny a
jsou odvozené ze zákona jejího zachování. Uvedeme pár dalších příkladů.

\subsection{Proudění v porézním prostředí}
Uvažujme porézní prostředí, kde podíl pórů v referenčním objemu je $\nu$ $[-]$. Tuto bezrozměrnou veličinu nazýváme porozita. 
Podíl tekutiny (vody) v referenčním objemu $\theta$ $[-]$ se nazývá saturace (opět bezrozměrná). Saturace se pohybuje od nějaké minimální (reziduální)
saturace $\theta_r$ po saturovaný podíl tekutiny $\theta_s$ obvykle rovný porozitě $\nu$. Pro tekutinu se zachovává její hmota, resp. hustota v prostoru
$\rho_V = \rho \theta$, 
kde $\rho$ je hustota tekutiny. V nejjednodušším případě uvažujeme nestlačitelnou kapalinu, plně saturovné porézní prostředí
a uvažujeme malé tlaky. V tom případě je $\rho$ i $\theta$ konstanta. Pak z obecné rovnice kontinuity dostaneme:
\[
    -\div \vc j = f,
\]
kde $\vc j$ $[kg/m^2/s]$ je hustota toku tekutiny  a $f$ $[kg/m^3/s]$ je hustota objemových zdrojů tekutiny. Podobně jako v případě tepla 
je nejjednodušší vztah pro $\vc j$ dán gradientem tlaku $p$ $[Pa]=[kgm^2/s]$ pomocí tzv. Darcyho zákona:
\[
    \vc j = -\rho k\tn K \grad p.
\]
Zde $k=\kappa/\mu$ je hydraulická vodivost daná permeabilitou $\kappa$ $[m^2]$, která je vlastností porézního média, a viskozitou $\mu$ $[Pa.s]$, 
která je vlastností tekutiny. Tenzor $\tn K$ je jednotkový v případě izotropního prostředí, ale v případě anisotrpního prostředí je to obecně symetrický 
pozitivně definitní tenzor. Pokud má porézní materiál nějak orientované mikroskopické kapiláry, bude v jednom směru mít větší vodivost než ve směrech kolmých. 
Obecně může mít materiál tři různé vodivosti ve třech různých směrech $k_x$, $k_y$, $k_z$ a nakonec tento materiál může být libovolně natočen v prostoru 
pomocí matice rotace $Q$:
\[
    \tn K = \tn Q^T \tn D \tn Q,\quad 
    \tn D = \begin{pmatrix}
                k_x     &0      &0\\
                0       &k_y    &0\\
                0       &0      &k_z                
            \end{pmatrix}
\]
Vodivosti v hlavních směrech jsou vlastní čísla matice $\tn K$, musí být kladné. Matice rotace $\tn Q$ je tvořena (ortogonálními) vlastními vektory.
Zde máme příklad anisotropie hydrulické vodivosti. Podobně existují materiály s anisotropní tepelnou vodivostí, nebo anisotrponí pevností etc.

Dále můžeme uvažovat stlačitelnou tekutinu, resp. stlačitelný materiál okolo pórů. Pro použití rovnice kontinuity pořebujeme spočítat derivaci 
hustotu hmoty tekutiny v prostoru podle času:
\[
    \prtl_t \rho_V = \Big(\frac{\prtl \rho}{\prtl p} + \frac{\prtl \theta}{\prtl p}\Big)\prtl_t p = S\prtl_t p.
\]
Veličina $S$ $[kg/m^3/Pa]$ se nazývá storativita a zahrnuje jak stlačitelnost tekutiny $\prtl_p \rho$ tak stlačitelnost prostředí $\prtl_p \theta$.
Pro nasycené, stlačitelné porézní prostředí tedy máme rovnici:
\[
    S \prtl_t p -\div\Big( \rho k\tn K \grad p\Big) = f
\]

Pro nenasycené prostředí pak dostáváme záporné (sací) tlaky $p$ a pro ně saturaci $\theta_r \le \theta(p) \le \theta_s$, která je funkcí tlaku. 
Navíc i vodivost $k$, klesá s klesajícím nasycením, je tedy $k(\theta)$ funkcí saturace. Dohromady dostaneme tzv. Richardsovu rovnici:
\[
    \prtl_t \theta(p) -\div\Big( \rho k(\theta(p)) \tn K \grad p\Big) = f
\]
kde funkce $\theta(p)$ a $k(\theta)$ jsou obecně nelineární a dostáváme tak nelineární parciální diferenciální rovnici.



\subsection{Transport chemických látek}
\begin{equation}
    \prtl_t (\rho_i c_i) + \div (\rho_i c_i \frac{\vc u}{\nu}) - \div( \tn D \grad c_i) = r_{ij}.
\end{equation}


\section{Vlnová rovnice (akustika)}

Odvodíme rovnici pro kmitání struny. Stav struny v čase $t$ a poloze $x$ je dán výchylkou struny $u(t,x)$. Pro zjendodušení si představujeme, 
že struna může kmitat jen v jednom směru. Na element daný intervalem $(a,b)$ působí síly v koncových bodech:
\[
    \vc F(t,a) = -T(t,a) \vc t(t,a),\quad \vc F(t,b) = T(t,b) \vc t(t, b)
\]
kde $T$ je napětí ve struně a $\vc t$ je tečný vektor $ t(t, x) = ( 1, \prtl_x u(t,x) )$. 
Jelikož v horizontálním směru se struna nepohybuje můsí být horizontální složka součtu sil rovna nule:
\[
    T(t,b) - T(t,a) = 0
\]
a jelikož jsme body $a$ a $b$ volili libovolně,  je napětí ve struně nazávislé na poloze: $T(t, x) = T(t)$.
Proto pro vertikální složku síly platí 
\[
    F_y = F_y(t,a) + F_y(t,b) = T(t)\big(\prtl_x u(t,b) - \prtl_x u(t,a)\big)
\]
Nyní použijeme 2. Newtonův zákon:
\[
    \frac{\d}{\dt} \int_a^b \rho(x) \prtl_t u(t,x) \d x = F_y(t,x) = T(t)\big(\prtl_x u(t,b) - \prtl_x u(t,a)\big) = T(t) \int_a^b \prtl_{xx} u(t,x) \d x 
\]
A jelikož $a$ a $b$ jsou libovolné, dostáváme bodovou rovnici:
\[
    \rho(x) \prtl_{tt} u(t,x) = T(t) \prtl_{xx} u(t,x) 
\]
Pokud předpokládáme konstantní hustotu $\rho(x) = \rho_0$ a zanedbáme změnu napětí struny při malé výchylce $T(t)=T_0$ dostaneme vlnovou rovnici ve tvaru:
\[
    \prtl_{tt} u(t,x) = c^2 \prtl_{xx} u(t,x)
\]
kde 
\[
    c=\sqrt{\frac{T_0}{\rho_0}}
\]
je rychlost šíření vlny. 
Mírně kompplikovanější je odvození vlnové rovnice pro změny (akustického) tlaku v prostoru:

\[
    \prtl_{tt} p(t, \vc x) = c^2 \Lapl p(t, \vc x)
\]
kde pro rychlost zvuku $c$ platí:
\[
    c = \sqrt{\frac{B}{\rho_0}},\quad B = \rho_0 \frac{\prtl P}{\prtl \rho}
\]
přičemž $B$ je objemová stlačitelnost při adiabatické expanzi. Pro vzduch máme $B=1.45\times 10^5\ Pa$ a hustotu $\rho_0 = 1.2kg/m^3$ a dostáváme rychlost zvuku:
\[
    c=347 \sqrt{\frac{kg.m/s^2/m^2}{kg/m^3}} = 1251 km/h
\]
Tabulková hodnota je $340m/s$.








\section{Mechanika}
\dots
\section{Elektromagnetismus}

\section{Klasifikace PDR}

\subsection{Eliptické rovnice}
Základním příkladem je Laplaceova rovnice:
\[
    \Lapl u(\vc x) = 0
\]
respektive Poisonova rovnice
\[
    \Lapl u(\vc x) = f(\vc x).
\]
Dalšími příklady je stacionární rovnice vedení tepla:
\[
    \div( k \grad T(\vc x) ) = f(\vc x),
\]
resp. stacionární rovnice Darcyho proudění:
\[
    \div( \tn K \grad p(\vc x)) = f(\vc x).
\]
Obecná rovnici druhého řádu:
\[
  \div\Big( \tn A \grad u(\vc x)\Big) + \vc b \grad u(\vc x) + c u(\vc x)= f(\vc x)
\]
je eliptická, pokud $\tn A$ je symetrická pozitivně definitní matice.

Pro eliptické rovnice platí (za jistých omezeních pro $\vc b$ a $c$) tzv. princip maxima. Pokud $u$ je řešením eplitické rovnice na oblasti $\Omega$
pak 
\[
    \max_{\vc x \in \Omega} u(\vc x) \le \max_{\vc x \in \prtl \Omega}  u(\vc x).
\]
Podobně pro minimum:
\[
    \min_{\vc x \in \Omega} u(\vc x) \ge \min_{\vc x \in \prtl \Omega}  u(\vc x).
\]


\subsection{Parabolické rovnice rovnice}
Příkladem je nestacionární rovnice vedení tepla:
\[
    \prtl_t T -\div(k \grad T) = f
\]
Vlastnosti řešení:
\begin{itemize}
 \item I zde platí princip maxima vzhledem k okrajové podmínce.
 \item Pokles řešení v čase:
    \[
        \max_{\vc x \in \Omega} u(t, \vc x)   \le \max_{\vc x \in \Omega}  u(s, \vc x),\quad \text{pro } t\ge s.
    \]
\end{itemize}

 Rovnice "zhlazuje" počáteční podmínku.
Nekonečná rychlost šíření změn.
\subsection{Hyperbolické rovnice}

Příklad je vlnová rovnice. 
Eulerovy rovnice. Neplatí princip maxima. 
Kvalitativní vlastnosti řešení:
\begin{itemize}
 \item Konečná rychlost šíření (vln).
 \item Reverzibilní v čase ($u(-t, \vc x )$ je též řešením).
 \item Nezhlazuje.
 \item Nesplňuje princip maxima, řešení se může akumulovat v bodě (náraz vlny na pobřeží).
\end{itemize}


