\section{Opakování plošných a křivkových integrálů}

\subsection{Křivkový integrál 1. druhu}
Jaká je hmotnost vlasu? Představme si natažený vlas a předpokládejme, že takto natažený má konstantní hustotu. 
Na vlasu si zavedeme souřednici $t$, $t=0$ je začátek vlasu $t=1$ je konec vlasu.
V konkrétním bodě $t$ na vlasu má vlas průřez $S(t)$. Pro malý přírůstek $\d t$ je hmotnost kousku vlasu $\d m = \rho S(t)\d t$.
Celková hmotnost pak je:
\begin{equation}
    \label{eq:hmotnost_vlasu}    
    m=\int_0^1 \rho S(t) \d t = \int_0^1 \rho_t(t) \d t,\quad \rho_t(t) = \frac{\d m}{\d t}= \rho S(t)
\end{equation}


kde $\rho_t$ je délková hustota. Když vlas pustíme, tak se trochu zkrátí a zkroutí do nějaké křivky v prostoru. Původní bod $t$ má nyní v prostoru polohu 
$\vc \phi(t)$. Tím zkroucením se změní průřezy $S$, ale nezmění se délková hustota, takže hmotnost opět spočteme podle \ref{eq:hmotnost_vlasu}.
Nyní si představme, že vlas je v tíhovém poli $f(\vc x)$, pro jednoduchost si představujeme, že tíha působí pouze v směru $z$ a má skalární velikost $f$,
která se ovšem mění ve všech směrech. Jaká na vlas působí celková síla? Pro natažený vlas podél osy $x$ máme, $\d F = f(\vc x) \d m$, a tedy:
\[
  F = \int_0^1 \rho_t(t) f[(t,0,0)] \d t
\]
a pro zkroucený vlas:
\[
   F=\int_k f \rho_t \d k = \int_0^1 f\big(\vc \phi(t)\big)\rho_t(t) \dt
\]
Nakonec si představme, že jde o zkamenělý vlas uvnitř skalního bloku, jehož hustota $\rho(\vc x)$ je známá pro každý bod $\vc x$. Přírůstek síly působící
pouze na ten zkamenělý vlas je $\d F = f(\vc x) \rho(\vc x) S(t) \d l$, kde 
\[
    \d l = \sqrt{ \d x^2 + \d y^2 + \d z^2 } = \sqrt{ (\phi_x')^2+(\phi_y')^2+(\phi_z')^2} \dt = \abs{\vc \phi'(t)} \dt
\]
 je přírůstek délky křivky pro přírůstek parametru $\d t$. Celková síla působící na vlas pak je:
\[
   F=\int_k f \rho \d k = \int_0^1 f\big(\vc \phi(t)\big)\rho( \vc \phi(t)) \abs{\vc \phi'(t)} \dt=
\]

Derivace $\vc \phi'(t)$ je tečný vektor ke křivce $k$ a jeho velikost je skutečný přírůstek v prostoru pro přírůstek $\d t$.
Tento typ integrálu nazýváme křivkový integrál 1. druhu ze skalárního pole $f$ podél křivky $k$, která je dána parametricky:
\[
   k:\ \{ \vc \phi(t);\ t\in (0,1)\}
\]
Integrál je vlastně definován pomocí substituce $\vc x = \vec \phi(t)$:
\begin{equation}
   \label{eq:curve_1_int}
   \int_k f(\vc x) \d k = \int_0^1 f\big(\vc \phi(t)\big)\, \abs{\vc \phi'(t)} \dt=
   \int_0^1 f(\phi_x,\phi_y,\phi_z) \sqrt{ (\phi_x')^2+(\phi_y')^2+(\phi_z')^2} \dt.
\end{equation}
Integrál 1. druhu můžeme aplikovat i na vektorové pole, ale výsledkem pak bude vektor. 

Další (fyzikální) příklady použití křivkového integrálu prvního druhu.
\begin{itemize}
 \item {\bf Moment síly} (vůči počátku), $M(\vc x) = \vc F(\vc x) \times \vc x$. Celkový moment na ohnutém drátu:
 \[
    M = \int_k \vc F(\vc x)\times \vc x \d k = \int_0^1 \vc F(\vc\phi(t))\times \vc\phi(t) \abs{\vc \phi'(t)} \d t
 \]
 \item \label{item:delka} {\bf Délka křivky} je integrál (1. druhu) ze skalárního pole $f(x,y,z)=1$, tj.
\[
   L=\int_\alpha^\beta \abs{\vc \phi'(t)} \dt,\quad 
\]
 \item {\bf Průměrná teplota} na poledníku $k$. Poledník je myšlená křivka na povrchu země a tiše předpokládáme, že je hladká.
 \[
    T = \frac{1}{L}\int_k T(\vc x) \d k = \frac{1}{L}\int_0^1 T(\phi(t)) \abs{\vc \phi'(t)} \d t,
 \]
 kde $L$ je (skutečná) délka poledníku (viz. bod \ref{item:delka}).
 \item {\bf Hmota křivky nebo plochy} je integrál (1. druhu) ze skalárního pole hustoty $\rho(x,y,z)$.
\[
        M=\int_k \rho(x,y,z) \d k
\]
\item {\bf Souřadnice těžiště} křivky je vektor $(T_x,T_y,T_z)$ integrálů (1. druhu) z vektoru skalárních funkcí 
       $x\rho(x,y,z)$, $ y \rho(x,y,z)$, $z\rho(x,y,z)$ dělený celkovou hmotou $M$. Např. pro plochu $S$:
\[
        \vc T=\frac{1}{M} \int_k \vc x \rho(\vc x) \d k
\]

\item {\bf Moment setrvačnosti} vzhledem k ose $o$ je integrál (1. druhu) ze skalární funkce
$f(\vc x)=r^2\rho(\vc x)$, kde $r$ je vzdálenost bodu $\vc x]$ od osy $o$. Ideální je transformovat křivku i osu tak aby osa 
byla jedna ze souředných os, např. pro $o$ totožnou s osou $z$ je
\[
    I_z = \frac{1}{M} \int_k (x_x^2 + x_y^2) \rho(\vc x) \d k
\]




\end{itemize}


\subsection{Křivkový integrál 2. druhu}
Ve vektorovém zápisu je integrál (2. druhu) z vektorového pole $\vc F$ podél křivky $k$:
\begin{equation}
\label{eq:curve_2_int}
 \int_k \vc F \cdot \vc t_k \d k = \int_0^1 \vc F( \vc \phi(s) ) \cdot \vc \phi'(s) \d k
\end{equation}
Zde je $\vc t_k$ tečný vektor. Přesněji pokud $\d k$ je velikost tečného vektoru jako pro integrál 1. druhu, tak $\vc t_k$, je vlastně jednotkový tečný vektor.
Ovšem stále je třeba výrazy vlevo v \eqref{eq:curve_1_int}, \eqref{eq:curve_2_int}, \eqref{eq:surf_1_int}, \eqref{eq:surf_2_int} jsou pouze symboly (zkratky), 
pro to co stojí vpravo. Pro některé druhy operací stačí manipulovat se zkratkami, ale někdy je potřeba se ponořit do definice.

Příklady:
\begin{itemize}
 \item {\bf Práce síly po křivce.} Integrál 2. druhu z vektorové funkce síly.
\end{itemize}




\subsection{Plošný integrál 1. druhu}
Podobně jako v případě křivku je plocha dána zobrazením $\vc \phi(u,v)$ z množiny $M\subset \Real^2$ do $\Real^3$. Normála $N$ k ploše v bodě daném parametry $(u,v)$, 
t.j, v bodě $\vc \phi(u,v)$ je dána vektorovým součinem tečných vektorů:
\[
   \vc N= \vc t_u \times \vc t_v,\quad \vc t_u = \frac{\prtl \vc \phi}{\prtl u},\quad \vc t_v=\frac{\prtl \vc \phi}{\prtl v}.
\]
Jednotková normála je pak $\vc n = \vc N /\abs{\vc N}$.

Integrál (1. druhu) ze skalárního pole $f$ přes plochu $S=\{\vc x = \vc \phi(u,v),\ (u,v) \in M\}$ je definován:
\begin{equation}
    \label{eq:surf_1_int}
   \int_S f \d S = \iint_M f(\vc \phi(u,v)) \abs{\vc N(u,v)} \d u\d v=
   \iint_M f(\phi_x,\phi_y,\phi_z) \sqrt{ (N_x)^2 + (N_y)^2 + (N_z)^2} \d u\d v, 
\end{equation}
Pozor, pokud plocha není rovina, tak normála a tudíž i její velikost jsou funkcí parametrů $u, v$, \cite{CVUT} str. 151.

\begin{itemize}
 \item Velikost povrchu $P$  plochy $M$ je integrál (1. druhu) ze skalárního pole $f(x,y,z)=1$, tj.
\[   
   P=\int_M 1 \d S = \int_M \abs{\vc n(u,v)} \d u\d v.
\]
\item {\bf Hmota plochy} je integrál (1. druhu) ze skalárního pole hustoty $\rho(x,y,z)$.
\[
        M=\int_S \rho(x,y,z) \d S
\]
{\bf Souřadnice těžiště plochy} je vektor $\vc T$ integrálů (1. druhu) z vektoru skalárních funkcí 
$ \vc x \rho( \vc x)$ dělený celkovou hmotou $M$. Např. pro plochu $S$:
\[
        \vc T=\frac{1}{M}\int_S \vc x\rho(\vc x) \d S
\]

{\bf Moment setrvačnosti} vzhledem k ose $o$ je integrál (1. druhu) ze skalární funkce $f(\vc x)=r^2\rho(\vc x)$, 
kde $r$ je vzdálenost bodu $[x,y,z]$ od osy $o$. Pro osu $z$:
\[
    I_z = \frac{1}{M} \int_S (x_x^2 + x_y^2) \rho(\vc x) \d S
\]

{\bf Práce síly po křivce.} Integrál 2. druhu z vektorové funkce síly.
{\bf Tok kapaliny skrze plochu za jednotkový čas.} Integrál 2. druhu z vektorového pole rychlosti.

\end{itemize}


\subsection{Plošný integrál 2. druhu}
Podobně lze integrál (2. druhu) vektorového pole $\vc F$ skrze plochu $S$ napsat:
\begin{equation}
    \label{eq:surf_2_int}
  \int_S \vc F\cdot \vc n \d S = \int_M \vc F( \vc \phi(u,v) ) \cdot \big( \prtl_u \vc \phi \times \prtl_v \vc \phi\big) \d u\d v
\end{equation}

\begin{itemize}
 \item Tento integrál má význam celkového toku pole skrz plochu. Například množství kapaliny, které proteče skrze plochu za jednotkový čas.
\end{itemize}




\subsection{Integrační věty: Stokesova, Gaussova, Greenova}

{\bf Greenova věta} (integrace per partes):
Pokud má oblast $V$ hranici $S$, pak pro hladká skalární pole $u$ a $v$ platí:
\[
        \int_V  \prtl_x u  v \d V = \int_S u v  n_x \d S - \int_V u \prtl_x v \d V
\]
kde $n_x$ je složka jednotkové normály. Odtud pro hladké vektorové pole $\vc v$ dostaneme:
\[
        \int_V (\grad u) \cdot \vc v \d V = \int_S u \vc v \cdot \vc n \d S - \int_V u\, \div \vc v \d V
\]

{\bf Gaussova věta}: Pro objem $V$ ohraničený plochou $S$ platí
\[
        \int_V \div \vc F \d V = \int_S \vc F\cdot \vc n \d S.
\]
Můžeme odvodit z Greenovy věty, použitím $u=1$ a $\vc v = \vc F$:
\[
    0=\int_V (\grad 1) \cdot \vc F \d V  = \int_S 1\vc F \cdot \vc n \d S - \int_S 1 \div \vc F  \d V
\]


{\bf Stokesova věta}: Pro plochu $S$ ohraničenou uzavřenou křivkou $k$ platí
\[
        \int_S \rot \vc F \cdot \vc n \d S = \int_k \vc F\cdot \vc t \d \vc k.
\]

Hranici oblasti $\Omega$ zapisujeme též jako $\prtl \Omega$. 

\section{Zákony zachování, věta o transportu}
Konzervativní veličina.
\begin{itemize}
 \item Zachování hmoty.
 \item Zachování hybnosti.
 \item Zachování momentu hybnosti.
 \item Zachování energie. Zachování vnitřní energie, tepla.
\end{itemize}

Natahovací pytlík s vodou se třpytkama. Hustota třpytek v bodě $\vc x$ v čase $t$ je $\rho(t,\vc x)$. Pytlík v čase $t$, je
oblast (otevřená jednoduše souvislá množina) $\Omega_t$, takže ho můžeme různě deformovat. Počet třpytek v pytlíku je pořád stejný:
\[
    \frac{\d}{\d t} \int_{\Omega_t} \rho(t,\vc x) \d \vc x =0.    
\]
Popis deformace v čase. Bod $\vc x_0$ v čase $0$ je přesunut do bodu $\vc x_t$ v čase $t$. 
\[
  \vc x_t = X(t, \vc x_0).
\]
Rychlostní pole pak je $\vc u(t, \vc x_t)= \prtl_t \vc X(t, \vc x_0)$.

\begin{theorem}[Reynolds transport theorem]
Nechť $q(t, \vc x)$ je hladká skalární funkce na na oblasti $\Omega_t$. Oblast $\Omega_t$ je dána hladkým zobrazením $X(t, \vc X)$
a počáteční oblastí $\Omega_0$:
\[
    \Omega_t = \{\vc x_t = \vc X(t, \vc x_0); \vc x_0\in \Omega_0\}.
\]
Pak platí:
\begin{equation}
    \frac{\d}{\d t} \int_{\Omega_t} q(t,\vc x_t) \d \vc x_t = \int_{\Omega_t} \prtl_t q + \div( q\vc u ) \d \vc x_t.
\end{equation}
\end{theorem}
\begin{proof}
Nechť $\chi_0$ je hladká "klobouková" funkce nulová mimo $\Omega_0$ a "skoro jednotková" uvnitř $\Omega_0$:
\[
    \chi_0(\vc x) = B( \dist( \vc x , \prtl \Omega_0) ),
\]
 kde vzdálenost $\dist$ je kladná uvnitř $\Omega_0$ a záporná vně. Funkce $B$ je nulová na $(-\infty, 0)$, $B=1$ 
 na $(\epsilon, \infty)$, a je hladká a rostoucí na $(0,\epsilon)$.
Tuto funkci necháme "unášet" rychlostním polem $\vc u$, takže se v čese $t$ zdeformuje:
\[
  \chi(t, \vc X(t, \vc x_0))=\chi_0(\vc x_0).
\]
Pro materiálovou derivaci funkce $\chi(t, \vc X)$ platí:
\[
  \frac{\d}{\d t} \chi(t, \vc X) = \frac{\d}{\d t} \chi(t, \vc X(t, \vc x_0))=\prtl_t \chi(t, \vc X) + \sum_i \prtl_{X_i} \chi(t, \vc X) \prtl_t X_i(t, \vc x_0)=
  \prtl \chi + \vc u \cdot \grad \chi 
\]
Nyní spočítáme :
\[ 
    \frac{\d}{\d t} \int_{\Omega_t} q \chi \d \vc x =
    \frac{\d}{\d t} \int_{\Real^3} q \chi \d \vc x =
    \int_{\Real^3} (\prtl_t q) \chi + q (\prtl_t \chi) \d \vc x =
    \int_{\Real^3} (\prtl_t q) \chi - q (\vc u \cdot \grad \chi) \d \vc x=
    \int_{\Omega_t} \big[ \prtl_t q + \div(q \vc u)\big] \chi \d \vc x
\]
Klobouková funkce $\chi$ může být libovolně blízko \emph{charakteristické funkci} oblasti $\Omega_t$ z čehož plyne důsledek věty.

\end{proof}


Např. zákon zachování hmoty můžeme napsat jako:
\[
    \frac{\d}{\d t} \int_{\Omega_t} \rho(t,\vc x) \d \vc x = \int_{\Omega_t} \prtl_t \rho + \div (\rho \vc u) \d \vc x = 0
\]
kde $\vc u$ je rychlost plynu a $\rho$ jeho hustota.

A jelikož toto platí pro libovolnou $\Omega_t$, pak pro hladké $\rho$ a $\vc u$ platí:
\[
    \prtl_t \rho + \div (\rho \vc u) =0
\]
což je \emph{rovnice kontinuity} pro hustotu stlačitelného plynu.

\subsection{Eulerovy rovnice}
Uvažujme materiál (tekutinu, nebo elastickou pevnou látku) s rychlostním polem  $u$. Ze zákona zachování hybnosti $\rho u$ 
plyne použitím rovnice kontinuity:
\[
    \prtl_t (\rho \vc u_i) +\div(\rho \vc u_i \vc u) = - \grad P,
\]
kde $P$ je tlak, a jeho záporný gradient je hustota síly, která způsobuje změnu hybnosti podle 2. Newtonova zákona.
Tato rovnice spolu s rovnicí kontinuity pro plyn:
\[
    \prtl_t \rho + \div( \rho \vc u)=0
\]
tvoří systém tzv. Eulerových rovnic popisujících proudění neviskózní stlačitelné tekutiny.


\section{Odvození rovnice vedení tepla}
Rovnice kontinuity platí za předpokladu, že "pohyb veličiny" $q$ je způsoben unášením v rychlostním poli $u$.
Přirozená interpretace je, že se jedná o rychlostní 
pole média, např. tekutiny. To ovšem obecně neplatí. Například pro koncentaci soli v roztoku platí také zákon zachování a sůl se  
pohybuje i v (makroskopicky) stacionárním objemu vody pomocí difúze. Je tedy třeba $u$ interpretovat jinak.

Definujeme plošný tok $\vc j$ veličiny $q$ jako množství veličiny, které projde jednotkovou elementární plochou za jednotku času. Tedy  
uvažujeme nekonečně malou plošku $\Delta S$ v bodě $\vc x$ s normálou $\vc n=\vc e_i$ (bázový vektor) a nekonečně malou zmenu 
času $\Delta t$.  Pokud mezi časy $t$ a $t+\Delta t$ projde skrz $\Delta S$ množství $\Delta Q$ veličiny $q$, platí
\[
    j_i(t, \vc x) = \frac{\Delta Q}{\Delta S \Delta t} 
\]
Pro pevnou oblast $\Omega$ je pokles množství veličiny $q$ v $\Omega$ roven celkovému toku veličiny ven z $\Omega$ přes její hranici:
\[
    -\frac{\d}{\d t} \int_\Omega q \d x = \int_{\prtl \Omega} \vc j \cdot \vc n \d S = \int_\Omega \div \vc j 
\]
Odtud dostaneme bodovou formu obecné rovnice kontinuity:
\[
    \prtl_t q + \div \vc j = 0.
\]
Ke stejnému výsledku dojdeme pokud použijeme Reynoldsovu větu pro rychlostní pole $\vc u = \vc j / q$.

Tok $\vc j(t,\vc x)$ je obecně nějakou funkcí závislou na lokálním chování veličiny $q$ na okolí bodu $(t, \vc x)$, může tedy záviset na 
$q$, $\grad q$, na $\prtl_t q$ a případně na vyšších prostorových a časových derivacích. 
Může také záviset na nějakých dalších veličinách, jako například na rychlosti média, viz. případ $\vc j = q\vc u$.

Nyní uvažujme specuálně zákon zachování pro energii pevného tělesa. Energie elementárního objemu $\Delta V$ je dána jeho teplotou jako:
\[
    \Delta E =  C \rho T\Delta V 
\]
kde $C$ $[J/K/kg]$ je tepelná kapacita a $T$ $[K]$ je teplota. Teplený tok $\vc j$ $[W/m^2]$ je v nejjednodušší podobě dán
Fourierovým zákonem:
\[
    \vc j = -k\grad T
\]
přičemž tepelná vodivost $k$ $[W/m/K]$ může být případně funkcí teploty $k(T)$. Dostáváme tak rovnici vedení tepla:

\[
   \prtl_t (C \rho T) - \div ( k \grad T) = 0
\]
Pokud budou v materiálu nějaké objemové teplné zdroje $f$ $[W/m^3]$ dostaneme:

\[
   \prtl_t (C \rho T) - \div ( k \grad T) = f
\]

Pokud by se jednalo o vedení tepla v kapalině, musíme do $\vc j$ zarnout i transport kapalinou:
\[
   \prtl_t (C \rho T) + \div (C \rho T \vc u) - \div( k \grad T) = f.
\]

