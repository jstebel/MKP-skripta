\documentclass{article}
\usepackage[czech]{babel}
\usepackage[utf8]{inputenc}
\usepackage{amsmath}

\renewcommand{\div}{\operatorname{div}}
\newcommand{\R}{\mathbb R}
\newcommand{\vc}[1]{\boldsymbol{#1}}
\def\d{{\rm{d}}}

\begin{document}

\title{Test č. 1 -- B}
\date{13. 12. 2021}
\author{}
\maketitle


\noindent

Napište slabou formulaci problému vedení tepla.\\
Tepelný tok je dán vztahem
\[
q(x) = -\sin(x) u'(x) + cu,
\]
kde $c=const$. Dále platí rovnice kontinuity
\[
q'(x) = f(x) \qquad \textrm{v }\Omega=[a,b].
\]
Definovány jsou následující okrajové podmínky
\begin{eqnarray*}
u(x) &=& 0 \qquad \textrm{v } x=a, \\
-\sin(x) u'(x) &=& q_N \qquad \textrm{v } x=b.
\end{eqnarray*}

\begin{enumerate}
\item Zakreslete oblast $\Omega$, zaveďte souřadný systém, vyznačte okrajové podmínky.
\item Formulujte okrajové podmínky.
\item Definujte prostory, ve kterých leží funkce $u,v$.
\item Odvoďte slabou formulaci, aplikujte okrajové podmínky.
\end{enumerate}

\end{document}
