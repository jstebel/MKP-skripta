\documentclass{article}
\usepackage[czech]{babel}
\usepackage[utf8]{inputenc}
\usepackage{amsmath}

\renewcommand{\div}{\operatorname{div}}
\newcommand{\R}{\mathbb R}
\newcommand{\vc}[1]{\boldsymbol{#1}}
\def\d{{\rm{d}}}

\begin{document}

\title{Test č. 1}
\date{13. 12. 2021}
\author{}
\maketitle


\noindent

Napište slabou formulaci ustáleného problému vedení tepla v kovové tyči (oblast $\Omega=[a,b]$).\\
Tepelný tok je dán Fourierovým zákonem
\[
    q(x) = -(x^4+5)u'(x), \qquad \textrm{v }\Omega,
\]
a platí rovnice kontinuity
\[
    q'(x) = \cos(3x) \qquad \textrm{v }\Omega.
\]

\noindent
Na pravé straně v bodě $b$ je udržována stálá teplota $U_b$.\\
Na druhé straně v bodě $a$ je udržován stálý vnější tepelný tok $q_a$.

\begin{enumerate}
\item Zakreslete oblast $\Omega$, zaveďte souřadný systém, vyznačte okrajové podmínky.
\item Formulujte okrajové podmínky.
\item Definujte prostory, ve kterých leží funkce $u,v$.
\item Odvoďte slabou formulaci, aplikujte okrajové podmínky.
\end{enumerate}

\vspace{2cm}
\emph{Bonus:}\\
Uvažujte slabou formulaci problému
\begin{align*}
    \sigma u(a)v(a) + \int\limits_{a}^{b} \lambda(x) & u'(x)v'(x) + u'(x)v(x) \,\d x = \\
        &\int\limits_{a}^{b} f(x)v(x) \,\d x - q_Nv(b) + \sigma u_R v(a) \qquad \forall v\in \mathrm{H}^1([a,b])
\end{align*}
Formulujte odpovídající silnou formulaci, tedy parc. diferenciální rovnici s okrajovými podmínkami.

\end{document}
