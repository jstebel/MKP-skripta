\documentclass{article}
\usepackage[czech]{babel}
\usepackage[utf8]{inputenc}
\usepackage{amsmath}

\renewcommand{\div}{\operatorname{div}}
\newcommand{\R}{\mathbb R}
\newcommand{\vc}[1]{\boldsymbol{#1}}
\def\d{{\rm{d}}}

\begin{document}

\title{Opravná zápočtová písemka}
\date{23. 1. 2020}
\author{}
\maketitle

\noindent
1) \hspace{0.5cm}
Napište slabou formulaci ustáleného problému difúze na oblast $\Omega=[a,b]$.\\
Difuzní tok je dán Fickovým zákonem
\[
    q(x) = -D(x)u'(x), \quad D(x)>0 \qquad \textrm{v }\Omega
\]
a platí bilanční rovnice
\[
    q'(x) = 1-x \qquad \textrm{v }\Omega.
\]
\noindent
V bodě $a$ je difuzní tok skrz membránu dán rozdílem koncentrací: $\sigma(u(a)-u_R)$, kde
$\sigma>0$ je přechodový koeficient a $u_R$ je stálá koncentrace udržovaná na vnější straně membrány.
V bodě $b$ je předepsán difuzní tok $q_N$.

\vspace{1cm}
\begin{enumerate}
\item Zakreslete oblast $\Omega$, zaveďte souřadný systém, vyznačte okrajové podmínky.
\item Definujte prostory, ve kterých leží funkce $u,v$.
\item Odvoďte slabou formulaci, aplikujte okrajové podmínky.
\end{enumerate}




\vspace{2cm}

\noindent
2) \hspace{0.5cm}
Určete, zda posloupnost funkcí
\[
f_n(x) = |x-1|^{\frac{1}{2n-1}}, \qquad x\in[0,2],
\]
konverguje ke své bodové limitě v normě $L_2$ a $L_\infty$.

\vspace{1cm}
\begin{enumerate}
\item Načtrněte graf několika prvních funkcí posloupnosti.
\item Určete bodovou limitu $f$.
\item Zapište symbolicky, co znamená, že $f_n$ konverguje k $f$ ve zvolené normě.
\item Vypočítejte.
\end{enumerate}




\end{document}
