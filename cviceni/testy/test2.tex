\documentclass{article}
\usepackage[czech]{babel}
\usepackage[utf8]{inputenc}
\usepackage{amsmath}

\renewcommand{\div}{\operatorname{div}}
\newcommand{\R}{\mathbb R}
\newcommand{\vc}[1]{\boldsymbol{#1}}
\def\d{{\rm{d}}}

\begin{document}

\title{Zápočtový test č. 2}
\date{12. 1. 2018}
\author{}
\maketitle

\begin{enumerate}
\item Napište slabou formulaci problému vedení tepla.\\
Tepelný tok je dán vztahem
\[
q(x) = -x^2 u'(x) + cu,
\]
kde $c=const$. Dále platí rovnice kontinuity
\[
q'(x) = f(x) \qquad \textrm{v }\Omega=[a,b].
\]
Definovány jsou následující okrajové podmínky
\begin{eqnarray*}
u(x) &=& 0 \qquad \textrm{v } x=a, \\
-x^2 u'(x) &=& q_N \qquad \textrm{v } x=b.
\end{eqnarray*}
Definujte prostory, ve kterých leží funkce $u,v$.

\item Aplikujte Galerkinovu metodu na slabou formulaci
\begin{equation*}
    a(u,v) = l(v) \qquad \forall v\in V=\{w\in H^1([a,b]), w(a)=0\},
\end{equation*}
kde $u\in V$ a dále
\begin{eqnarray*}
    a(u,v) &=& \int_a^b \Big( Ku'(x)v'(x) + g(x)u(x)v(x) \Big) \d x, \\
    l(v) &=& \int_a^b f(x)v(x) \d x - q_N v(b).
\end{eqnarray*}

Použijte lineární konečné prvky, definujte diskrétní slabé řešení a příslušný diskrétní vektorový prostor.
Popište důkladně vaši zvolenou diskretizaci (indexy, počty uzlových bodů, elementů stupňů volnosti atd.), 
použijte i vhodný obrázek.
Sestavte příslušný lineární algebraický systém, uveďte lokální matice a jejich pravé strany.

\item
Pro bilineární formu $a$ z úlohy 2 definujte omezenost a elipticitu.\\
Pokuste se dokázat, že $a$ je skutečně eliptická (doplňte nutné předpoklady o $K$ a $g(x)$).

\end{enumerate}

\end{document}
