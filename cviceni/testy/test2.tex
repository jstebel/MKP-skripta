\documentclass{article}
\usepackage[czech]{babel}
\usepackage[utf8]{inputenc}
\usepackage{amsmath}

\renewcommand{\div}{\operatorname{div}}
\newcommand{\R}{\mathbb R}
\newcommand{\vc}[1]{\boldsymbol{#1}}

\begin{document}

\title{Zápočtový test č. 1}
\date{11. 12. 2017}
\author{}
\maketitle

\begin{enumerate}
\item Napište slabou formulaci problému vedení tepla.\\
Tepelný tok je dán vztahem
\[
q(x) = -(3+x)^2 u'(x)
\]
a rovnice kontinuity
\[
q'(x) = x + \sigma(u_f - u(x)) \qquad \textrm{v }\Omega=[a,b],
\]
kde $\sigma>0$ a $u_f$ jsou konstanty.\\
Definovány jsou následující okrajové podmínky
\begin{eqnarray*}
-q(x) &=& q_N \qquad \textrm{v } x=a, \\
u(x) &=& u_b \qquad \textrm{v } x=b.
\end{eqnarray*}
(Definujte příslušné vektorové prostory a vhodné bilineární a lineární formy.)


\item Aplikujte Galerkinovu metodu na slabou formulaci
\begin{multline*}
    \sigma u(a)v(a) + \int\limits_a^b k(x) u'(x)v'(x) + cu'(x)v(x) \,\textrm{d}x =\\
    = \int\limits_a^b f(x)v(x) \,\textrm{d}x + \sigma u_r v(a) \\ \forall v\in V=\{w\in H^1(\Omega), w(b) = 0\},\; u\in V.
\end{multline*}
Použijte lineární konečné prvky, definujte diskrétní slabé řešení a příslušný diskrétní vektorový prostor.
Sestavte příslušný lineární algebraický systém, uveďte lokální matice a jejich pravé strany.

\emph{Bonus: Nalezněte příslušnou silnou formulaci -- parciální differenciální rovnici včetně okrajových podmínek.}

\end{enumerate}

\end{document}
