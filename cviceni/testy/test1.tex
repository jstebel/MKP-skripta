\documentclass{article}
\usepackage[czech]{babel}
\usepackage[utf8]{inputenc}
\usepackage{amsmath}

\renewcommand{\div}{\operatorname{div}}
\newcommand{\R}{\mathbb R}
\newcommand{\vc}[1]{\boldsymbol{#1}}

\begin{document}

\title{Zápočtový test č. 1}
\date{10. 11. 2016}
\author{}
\maketitle

\begin{enumerate}
\item Ocelový válec o výšce $l$ a poloměru $r$ je na jedné podstavě zahříván tepelným tokem $g=10$ $W/m^2$
a na druhé chlazen na teplotu $T_0=280$ $K$.
Plášť válce je tepelně izolován.
Tepelná vodivost oceli je $k=80$ $W/m/K$, tepelná kapacita $C=450$ $J/K/kg$, hustota $\varrho=7800$ $kg/m^3$.
Popište ustálený proces vedení tepla (tj. s teplotou nezávislou na čase) jako eliptickou parciální diferenciální rovnici s okrajovými podmínkami.

\item Odvoďte slabou formulaci pro eliptickou okrajovou úlohu:
\[ -\Delta u + u = f \mbox{ v }\Omega,\quad \nabla u\cdot\vc n=g-u \mbox{ na }\partial\Omega \]
(využijte toho, že $\Delta u=\div(\nabla u)$).

\end{enumerate}

\end{document}